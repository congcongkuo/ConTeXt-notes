\startcomponent metafun
\product ctxnotes

\chapter{我是这样学习 MetaFun 的}

其实,我本来是想尽快制作一份有关 \CONTEXT\ 演示文档的学习笔记,尝试了几次之后,都铩羽而归。我渐渐意识到如果不了解 \METAFUN,那么使用 \CONTEXT\ 是很痛苦的,就像拿到了一座金山却无处消费它一般。我开始认真学习 \METAFUN,并且也很后悔当初没有学习 \METAPOST,而在 PGF/TikZ 上却曾经浪费了很多时间。

\section{从画一个矩形开始}

在这份这份文档的封面上,画有许多形态各异的矩形,它们不是我画的,而是我从一份 \CONTEXT\ 演示文档模板中扣出来的一段代码。我从现在开始逐步研究它们是如何绘制出来的。

先进行一个矩形区域的填充:

\startniceTEX[numbering=line]
\setupcolors[state=start]
\setupcolor[hex]
\definecolor[BackgroundColor][h=cccccc]

\starttext
\startuseMPgraphic{box}
offset := uniformdeviate 10pt;
width  := 2*offset + 30pt + uniformdeviate 30pt;
height := 2*offset + 20pt + uniformdeviate 10pt;
x1 := offset ; y1 := offset; x2 := width-offset; y2 := height-offset;
fill z1--(x2,y1)--z2--(x1,y2)--cycle 
    withcolor \MPcolor{BackgroundColor};
\stopuseMPgraphic

\useMPgraphic{box}
\stoptext
\stopniceTEX

\definecolor[BackgroundColor][h=cccccc]
\startuseMPgraphic{box}
offset := uniformdeviate 10pt;

width  := 2*offset + 30pt + uniformdeviate 30pt;
height := 2*offset + 20pt + uniformdeviate 10pt;
x1 := offset ; y1 := offset; 
x2 := width-offset; y2 := height-offset;

fill z1--(x2,y1)--z2--(x1,y2)--cycle 
    withcolor \MPcolor{BackgroundColor};
\stopuseMPgraphic

这个小例子可在页面中绘制一个灰色的矩形填充域\;\useMPgraphic{box}\;\;。实际上,单单要完成这个例子所实现的目标,没必要去定义那么多的变量,这里之所以如此,不妨认为我想借此来熟悉一下 \METAPOST\ 的变量赋值与运算语法。

代码第 1-3 行的大意是启用颜色环境,并将颜色描述格式设为 16 进制码形式,最后使用灰色定义了一种名曰 BackgroundColor 的颜色;在 \METAFUN\ 中,我们可以使用 \tex{MPcolor} 命令将 \CONTEXT\ 中定义的颜色转换为 \METAFUN\ 所使用的颜色值,详见代码第 12 行。这多少可以体现出 \METAFUN\ 的一些特点,它实现了 \CONTEXT\ 与 \METAPOST\  的沟通。

代码第 7 行演示了 uniformdeviate 命令的用法,它的作用就是从 \type{[0pt - 10pt]} 这个尺寸区间中随机取一个尺寸。

代码第 11-12 行演示了如何填充一条路径,比较值得注意的是路径构成中的 \type{z1} 与 \type{z2} 的含意,这似乎是 \METAPOST\ 约定的,它们分别表示 \type{(x1,y1)} 与 \type{(x2,y2)},但愿我没有理解错。

代码第 6-13 行构造了一幅 \METAFUN\ 图形,第 15 行代码将该图形插入至文档中。这就是 \METAFUN\ 图形嵌入于 \CONTEXT\ 的一般方式。另外,也可以像插入普通图片那样:

\startniceTEX
\placefigure
  [here,force][fig:box]
  {嵌入 \METAFUN\ 图形}{\useMPgraphic{box}}
\stopniceTEX

\placefigure
  [here,force][fig:box]
  {嵌入 \METAFUN\ 图形}{\useMPgraphic{box}}

下面为这个矩形填充域绘制边框。如果只是绘制一个矩形框,这比较简单,以下示例便可以:

\startniceTEX
\setupcolors[state=start]
\setupcolor[hex]
\definecolor[BackgroundColor][h=cccccc]
\definecolor[OrnamentColor][h=99ccff]

\starttext
\startuseMPgraphic{box-ornament}
offset := uniformdeviate 10pt;
width  := 2*offset + 30pt + uniformdeviate 30pt;
height := 2*offset + 20pt + uniformdeviate 10pt;
x1 := offset ; y1 := offset; x2 := width-offset; y2 := height-offset;
fill z1--(x2,y1)--z2--(x1,y2)--cycle 
    withcolor \MPcolor{BackgroundColor};
draw z1--(x2,y1)--z2--(x1,y2)--cycle
    withcolor \MPcolor{OrnamentColor};
\stopuseMPgraphic

\useMPgraphic{box-ornament}
\stoptext
\stopniceTEX

\definecolor[BackgroundColor][h=cccccc]
\definecolor[OrnamentColor][h=99ccff]
\startuseMPgraphic{box-ornament}
offset := uniformdeviate 10pt;
width  := 2*offset + 30pt + uniformdeviate 30pt;
height := 2*offset + 20pt + uniformdeviate 10pt;
x1 := offset ; y1 := offset; x2 := width-offset; y2 := height-offset;
fill z1--(x2,y1)--z2--(x1,y2)--cycle 
    withcolor \MPcolor{BackgroundColor};
draw z1--(x2,y1)--z2--(x1,y2)--cycle
    withpen pencircle scaled 2pt withcolor \MPcolor{OrnamentColor};
\stopuseMPgraphic

\noindent 得到\;\useMPgraphic{box-ornament}\;,但是它无法实现本文档封面页上的那些小矩形框的效果,它们是无法采用一条路径来实现,而是要四个边框逐个的画:

\startniceTEX[numbering=line]
... ...
def delta = ((uniformdeviate .5offset) + .25offset) enddef;
drawoptions(withpen pencircle scaled 2pt 
    withcolor \MPcolor{OrnamentColor});
draw (x1-delta,y1)--(x2+delta,y1);
draw (x2,y1-delta)--(x2,y2+delta);
draw (x2+delta,y2)--(x1-delta,y2);
draw (x1,y2+delta)--(x1,y1-delta);

drawoptions();
setbounds currentpicture to unitsquare xyscaled (width,height);
\stopniceTEX

\startuseMPgraphic{box-random-ornament}
offset := uniformdeviate 10pt;
def delta = ((uniformdeviate offset) + .25offset) enddef;
width  := 2*offset + 30pt + uniformdeviate 30pt;
height := 2*offset + 20pt + uniformdeviate 10pt;
x1 := offset ; y1 := offset; x2 := width-offset; y2 := height-offset;
fill z1--(x2,y1)--z2--(x1,y2)--cycle 
    withcolor \MPcolor{BackgroundColor};
drawoptions(withpen pencircle scaled 2pt 
    withcolor \MPcolor{OrnamentColor});
draw (x1-delta,y1)--(x2+delta,y1);
draw (x2,y1-delta)--(x2,y2+delta);
draw (x2+delta,y2)--(x1-delta,y2);
draw (x1,y2+delta)--(x1,y1-delta);

drawoptions();
setbounds currentpicture to unitsquare xyscaled (width,height);
\stopuseMPgraphic

\noindent 这样才可以得到那种类似于手绘矩形框的效果\;\useMPgraphic{box-random-ornament}\;\;。注意,第 2 行代码所定义的 \type{delta} 似乎是一个变量,实际它是一个宏,一个不带任何参数的宏。宏可以避免许多重复的输入,并且使得代码更为易读;\type{drawoptions} 也具有类似作用。在上述代码片段中,宏还起到了让 \type{delta} 返回的尺寸大小居于随机变化的效果,因为宏在被 \METAPOST\ 处理时要被其定义替换,因为 \type{delta} 定义中的 \type{uniformdeviate} 运算符就被反复调用,生成不同的随机数。\type{setbounds} 语句用于设置图形的边界,这样可以保证将所绘制的图形完全地显示出来(其实我觉得这有点多余)。

下面,来看一下带参数的宏该如何写。带参数的宏,可将它理解为可重复调用的函数。对于前面的示例,可以将这一系列的绘制过程定义为下面的宏:

\startniceTEX
def random_box (expr width, height, offset, linewidth ) =
    def delta = ((uniformdeviate .5offset) + .25offset) enddef;

    x1 := offset; y1 := offset; x2 := width-offset; y2 := height-offset;

    drawoptions(withcolor transparent(1,.8,\MPcolor{BackgroundColor}));
    fill z1--(x2,y1)--z2--(x1,y2)--cycle;

    drawoptions(withpen pencircle scaled linewidth 
        withcolor \MPcolor{OrnamentColor});
    draw (x1-delta,y1)--(x2+delta,y1);
    draw (x2,y1-delta)--(x2,y2+delta);
    draw (x2+delta,y2)--(x1-delta,y2);
    draw (x1,y2+delta)--(x1,y1-delta);

    drawoptions();
    setbounds currentpicture to unitsquare xyscaled (width,height);
enddef;
\stopniceTEX

\type{random_box} 宏是可以接受传入参数的。\METAPOST\ 宏所支持的参数类型有:
\type{expr}、\type{text} 与 \type{suffix}。\type{expr} 参数类型的用法正如上例 \type{random_box} 宏定义所示,它表示宏的使用者可以向其传入表达式的值。至于 \type{text} 与 \type{suffix} 参数类型,由于现在尚未用到它们,所以我可以暂时不理睬它们的功用。

下面演示如何使用上述定义的 \type{random_box} 宏:

\startniceTEX
offset := uniformdeviate 10pt ;
width  := 2*offset + 40pt + uniformdeviate 30pt ;
height := 2*offset + 30pt + uniformdeviate 10pt ;

random_box (width,height,offset,1pt);
\stopniceTEX

\section[ordinary-repeat]{平凡的重复}

一个士兵会踢正步,实在太平凡了;一群士兵会踢同样节奏的正步,就开始不平凡起来了。平凡的重复,往往是不平凡的。现在,我们开始重复上一节所绘制的那个 Box。先来看下面这个可编译的示例:

\startniceTEX[numbering=line]
\startuseMPgraphic{random-box}
def random_box (expr width, height, offset, linewidth ) =
    def delta = ((uniformdeviate .5offset) + .25offset) enddef;

    x1 := offset; y1 := offset; x2 := width-offset; y2 := height-offset;

    drawoptions(withcolor \MPcolor{BackgroundColor});
    fill z1--(x2,y1)--z2--(x1,y2)--cycle;

    drawoptions(withpen pencircle scaled linewidth 
        withcolor \MPcolor{OrnamentColor});
    draw (x1-delta,y1)--(x2+delta,y1);
    draw (x2,y1-delta)--(x2,y2+delta);
    draw (x2+delta,y2)--(x1-delta,y2);
    draw (x1,y2+delta)--(x1,y1-delta);

    drawoptions();
    setbounds currentpicture to unitsquare xyscaled (width,height);
enddef;

for i=1 upto 400 :
    offset := uniformdeviate 4pt ;
    width  := 2*offset + 10pt + uniformdeviate 10pt ;
    height := 2*offset + 5pt + uniformdeviate 4pt ;

    addto currentpicture also
        image(random_box (width,height,offset,1pt)) shifted
            (uniformdeviate 8cm, uniformdeviate 6cm) ;
endfor ;
\stopuseMPgraphic
\stopniceTEX

\startuseMPgraphic{random-box}
def random_box (expr width, height, offset, linewidth ) =
    def delta = ((uniformdeviate .5offset) + .25offset) enddef;

    x1 := offset; y1 := offset; x2 := width-offset; y2 := height-offset;

    drawoptions(withcolor \MPcolor{BackgroundColor});
    fill z1--(x2,y1)--z2--(x1,y2)--cycle;

    drawoptions(withpen pencircle scaled linewidth 
        withcolor \MPcolor{OrnamentColor});
    draw (x1-delta,y1)--(x2+delta,y1);
    draw (x2,y1-delta)--(x2,y2+delta);
    draw (x2+delta,y2)--(x1-delta,y2);
    draw (x1,y2+delta)--(x1,y1-delta);

    drawoptions();
    setbounds currentpicture to unitsquare xyscaled (width,height);
enddef;

for i=1 upto 400 :
    offset := uniformdeviate 4pt ;
    width  := 2*offset + 10pt + uniformdeviate 10pt ;
    height := 2*offset + 5pt + uniformdeviate 4pt ;

    addto currentpicture also
        image(random_box (width,height,offset,1pt)) shifted
            (uniformdeviate 8cm, uniformdeviate 6cm) ;
endfor ;
\stopuseMPgraphic

\noindent 我得到如下图所示的图形:

\placefigure
  [here,force][fig:random-box]
  {平凡的重复}{\useMPgraphic{random-box}}

第 26-28 行是解决 \type{random_box} 重复绘制的核心代码。\type{addto ... also} 是 \METAPOST\ 内部定义的宏,它可以将一幅图形 (picture) 添加到另一幅图形中。图形是 \METAPOST\ 中的一个数据结构,它可以存储多条已被绘制的路径,并且图形与图形之间是可以包容的。\type{image} 也是 \METAPOST\ 内部定义的宏,它可以将某一被绘制的路径转变为图形。至于 \type{shifted},顾名思义,它是用来进行几何平移变换的,而且变换的位置是随机坐标来确定的。

\section{Overlay 与背景}

Overlay 这是 \CONTEXT\ 的一个术语,它的名字取的不是很正确,其本意是“覆盖,覆盖图”,但是在 \CONTEXT\ 中,它通常是位于文本之下的图形,或者说它是被文本覆盖。不过,overlay 可以分为许多个层次,因此这个术语的名字取得也不是完全不正确。

很多时候,我都是将 overlay 作为一个背景层来对待的,下面的这个例子便可以说明这个问题:

\startniceTEX[numbering=line]
\setupcolors[state=start]

\starttext
\defineoverlay[tea][{\darkgreen\ss\bfc 绿茶}]
\framed[background=tea, align=middle]{今\\天\\出\\售}
\stoptext
\stopniceTEX

\defineoverlay[tea][{\darkgreen\ss\bfd 绿\;茶}]

\noindent 这个例子可以将“绿茶”作为“今天出售”的背景:\framed[background=tea, align=middle]{今\\天\\出\\售}\;\;。

\tex{defineoverlay} 的命令使用起来很简单,第一个参数是 overlay 名,第二个参数是 overlay 的内容,通常是一组排版或绘图命令的集合,推荐使用 \type{{}} 将其括起来。

在上述示例中,已经演示了使用文本作为 overlay 内容,下面来看一下使用图片的例子:

\startniceTEX
\defineoverlay[cow][{\externalfigure[cow]%
    [width=\overlaywidth,height=\overlayheight]}]
\framed[width=3cm,height=2cm,background=cow,align=right]
       {\vfill\color[darkred]{这是一头奶牛}}
\stopniceTEX

\defineoverlay[cow][{\externalfigure[cow][width=\overlaywidth,height=\overlayheight]}]

\noindent 于是我们便拥有了一头漂亮的奶牛 \framed[width=3cm,height=2cm,background=cow,align=right]{\vfill\color[darkred]{这是一头奶牛}} 。

上述以图片作为 overlay 的代码没必要多费口舌,多看几遍,动手尝试一下便可以明了。值得注意的倒是 \tex{framed} 命令的 \type{align} 参数,它的取值有些另类,明明是左对齐,但是它的值却是 right;如果你想让文本右对齐,就不得不将其设为 left。这实际上是 Hans 犯得一个错误,因为他当年在写 \CONTEXT\ 时,满脑子是 ragged left(左边不齐) 与 ragged right(右边不齐)。这样的错误一旦犯下了,便不好再改,主要是因为无法兼容过去的文档,后来 Hans 又加入了 flushleft 与 flushright 参数值,这是与我们普遍的习惯是相符的。

我们还可以将 \METAFUN\ 绘制的图形作为 overlay,这也是本章一个非常重要的内容,它是 \METAFUN\ 与 \CONTEXT\ 集成的基础,来看这个示例:

\startniceTEX
\startuseMPgraphic{demo circle}
  path p ;
  p := fullcircle xscaled \overlaywidth yscaled \overlayheight ;
  fill p withcolor \MPcolor{BackgroundColor};
  draw p withpen pencircle scaled 2pt withcolor \MPcolor{OrnamentColor};
\stopuseMPgraphic

\defineoverlay[demo circle][\useMPgraphic{demo circle}]
\framed[background=demo circle,frame=off]{被椭圆包围的文本}
\stopniceTEX

\startuseMPgraphic{demo circle}
  path p ;
  p := fullcircle xscaled \overlaywidth yscaled \overlayheight ;
  fill p withcolor \MPcolor{BackgroundColor};
  draw p withpen pencircle scaled 2pt withcolor \MPcolor{OrnamentColor};
\stopuseMPgraphic
\defineoverlay[demo circle][\useMPgraphic{demo circle}]

\noindent 我们可以得到 \framed[background=demo circle,frame=off,height=22pt]{被椭圆包围的文本}\;。

由上例可见,我们发现 \METAFUN\ 可以将 \CONTEXT\ 中的一些尺寸,比如 \tex{overlaywidth} 与 \tex{overlayheight} 等参数传入 \METAFUN\ 环境中。在本章开始的时候,还用过一个 \tex{MPcolor} 宏,可以将 \CONTEXT\ 的颜色定义转换为 \METAPOST\ 的颜色定义。我们还可以通过使用 \tex{the} 作为前缀,实现在 \METAFUN\ 环境中访问诸如 \tex{textwidth}、\tex{textheight}、\tex{topspace} 等尺寸参数。

前面说过,overlay 是可以叠加的,下面再来用一个例子演示一下:

\startniceTEX
\startuseMPgraphic{demo ellipse}
  path p ;
  p := fullcircle xscaled (.95*\overlaywidth) yscaled (.95*\overlayheight);
  fill p withcolor \MPcolor{BackgroundColor};
  draw p withpen pencircle scaled 2pt withcolor \MPcolor{OrnamentColor};
\stopuseMPgraphic

\startuseMPgraphic{demo square}
  path p ;
  p := fullsquare xscaled \overlaywidth yscaled \overlayheight;
  fill p withcolor \MPcolor{OrnamentColor};
\stopuseMPgraphic

\defineoverlay[demo ellipse][\useMPgraphic{demo ellipse}]
\defineoverlay[demo square][\useMPgraphic{demo square}]

\framed[background={demo square,demo ellipse},frame=off]{被矩形与椭圆包围的文本}
\stopniceTEX

\startuseMPgraphic{demo ellipse}
  path p ;
  p := fullcircle xscaled \overlaywidth yscaled \overlayheight;
  fill p withcolor \MPcolor{BackgroundColor};
  draw p withpen pencircle scaled 2pt withcolor \MPcolor{OrnamentColor};
\stopuseMPgraphic

\startuseMPgraphic{demo square}
  path p ;
  p := fullsquare xyscaled (1.1*\overlaywidth,1.1*\overlayheight);
  fill p withcolor \MPcolor{OrnamentColor};
\stopuseMPgraphic

\defineoverlay[demo ellipse][\useMPgraphic{demo ellipse}]
\defineoverlay[demo square][\useMPgraphic{demo square}]

\noindent 我们可以得到\quad\framed[background={demo square,demo ellipse},frame=off,height=22pt]{被矩形与椭圆包围的文本} 。这其实没什么技巧,只需多定义几个 overlay,然后将它们作为 \tex{framed} 的 \type{background} 的参数列表并按照先后的显示次序排列即可。

本节所举的例子,只是使用 \tex{framed} 命令来演示。事实上,\CONTEXT\ 有许多命令都支持 overlay,只要它具备 \type{background} 参数。

\section{代码的分离}

有时候,会希望能够在多个 \METAFUN\ 图形环境中引用同样的一段代码,在我不知道存在 \type{MPinclusions} 环境时,我写过下面这般风格的代码:

\startniceTEX
\startuseMPgraphic{TitlePage}
def random_box (expr width, height, offset, linewidth ) =
    ... ... ...
enddef

for i=1 upto 400 :
    ... ... ...
    addto currentpicture also
        image(random_hash_frame(width,height,offset,1pt)) shifted
            (uniformdeviate OverlayWidth, uniformdeviate OverlayHeight) ;
endfor ;
\stopuseMPgraphic

\startuseMPgraphic{PageNumberBox}
def random_box (expr width, height, offset, linewidth ) =
    ... ... ...
enddef

offset := uniformdeviate 10pt ;
random_box (\overlaywidth, \overlayheight, offset, 1pt);
\stopuseMPgraphic
\stopniceTEX

\noindent 虽然这样的代码可以用,但是它很脏。比如,如果我想修改 \type{random_box} 宏的定义,就意味着这两个图形环境中的 \type{random_box} 都要修改。当我我在 \METAFUN\ 手册中发现了 \type{MPinclusions} 这个环境时,立刻就将上述的代码格式修改为:

\startniceTEX

\startMPinclusions
def random_box (expr width, height, offset, linewidth ) =
    ... ... ...
enddef
\stopMPinclusions

\startuseMPgraphic{TitlePage}
for i=1 upto 400 :
    ... ... ...
    addto currentpicture also
        image(random_hash_frame(width,height,offset,1pt)) shifted
            (uniformdeviate OverlayWidth, uniformdeviate OverlayHeight) ;
endfor ;
\stopuseMPgraphic

\startuseMPgraphic{PageNumberBox}
offset := uniformdeviate 10pt ;
random_box (\overlaywidth, \overlayheight, offset, 1pt);
\stopuseMPgraphic
\stopniceTEX

\noindent 可以看出,\type{MPinclusions} 环境的作用就是用于存放各 \METAFUN\ 图形的公共代码,类似于 C 程序的 .h 文件。

我们还可以在 \METAFUN\ 图形环境中使用 \tex{includeMPgraphic} 来引用其它的 \METAFUN\ 图形:

\startniceTEX
\setupcolors[state=start]

\starttext
\startuseMPgraphic{test}
    fill fullsquare rotated 45 scaled 4cm withcolor \MPcolor{yellow} ;
\stopuseMPgraphic

\startuseMPgraphic{demo}
    \includeMPgraphic{test}
    fill fullsquare scaled 2cm withcolor \MPcolor{red} ;
\stopuseMPgraphic

\useMPgraphic{demo}
\stoptext
\stopniceTEX

\section{还有一些图形环境}

除了 \type{useMPgraphic} 环境之外,还有 \type{uniqueMPgraphic}、\type{MPpage} 图形环境,当然还有其它类型的,只是我目前认为掌握这三种,基本上够用了。

\type{uniqueMPgrahpic} 环境与 \type{useMPgraphic} 环境的区别是:前者只生成一份图形,可供多次使用;后者是在每一次使用时,都要生成图形。如果是要实现随机图形效果,也就是说多次使用同一份绘图代码生成的图形却不相同,这就只好使用 \type{useMPgraphic} 环境,这个问题在 \in[ordinary-repeat] 节中有所体现。

\type{MPpage} 环境可专门用于生成单独的矢量图形,试验一下这个例子便可体验其用处:

\startniceTEX
\setupcolors[state=start]
\setupMPpage
  [offset=1pt,
   background=color,
   backgroundcolor=gray]

\startuseMPgraphic{test}
  fill fullsquare rotated 45 scaled 4cm withcolor \MPcolor{yellow} ;
\stopuseMPgraphic

\starttext
\startMPpage
    \includeMPgraphic{test}
    fill fullsquare scaled 2.5cm withcolor \MPcolor{red} ;
\stopMPpage
\stoptext
\stopniceTEX

\section{后记}

这篇文章的内容,支离破碎。这是自然的。因为 \METAFUN\ 所涵括的内容远非我这一篇小文所能覆盖得了的。我以为 \METAFUN\ 手册最好是能通读一遍,不过我现在做不到,只好用什么,看什么。在这用与看的过程中,我想尽量地建立系统的认识。这是一个长期的过程。我唯一能做的就是尽量寻找可以使用 \METAFUN\ 的机会。

我觉得使用 \CONTEXT\ 排版文档,如果能够充分结合 \METAFUN\ 图形环境来设计文档版面的外观,这是非常具有挑战性的。个人的审美观以及绘制图形的能力决定了文档样式能否体现出个性。

\stopcomponent