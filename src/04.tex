\chapter[post]{散文}

一旦解决了 \CONTEXT\ 对汉字的支持问题,便可以尝试让一篇文章有它该有的样子,然后会觉得 \CONTEXT\ 越来越有用处。

一篇文章,它应该是什么样子呢?至少要有标题,有作者信息,还可能有次标题,次次标题……还要有段落,有页码。有了这些,足以用于记事。至于科技工作者通常所需要的列表、表格、数学公式、插图等排版元素,需要在文章这些该有的样子的基础上进一步构建,现在不必急于探求。

\section{标题}

在 \CONTEXT\ 中,标题分为两种,无编号的和有编号的。每种标题又分为诸多级别\index[zjbt]{章节标题}。无编号的标题,级别从高到低,排版命令依次为

\startTEX
\title{...} % 一级标题
\subject{...} % 次级标题
\subsubject{...} % 次次级标题
\subsubsubject{...} % 次次次级标题
... ... ...
\stopTEX

\noindent 有编号的标题,级别从高到低,排版命令依次为

\startTEX
\chapter{...} % 一级标题
\section{...} % 次级标题
\subsection{...} % 次次级标题
\subsubsection{...} % 次次次级标题
... ... ...
\stopTEX

\noindent 应该不难看出两种标题各自的次级标题降级规律。不建议使用级别层次太深的标题,否则会让读者觉得身陷迷宫,通常前三级标题足够使用。若是写一篇散文,标题只需要用 \type{\title}。若是写一本小说,只需用 \type{\title} 制作书名,用 \type{\chapter} 制作章名。我所写的这份文档,标题的层级也只是到次级标题。

\section[essay]{写一篇散文}

例 \in[zaoshu] 设定了段落首行缩进距离,用 \type{\title} 创建了文章标题。

\startTEX
\definefallbackfamily[myfonts][rm][latinmodernroman]
                     [range={0x0000-0x0400},force=yes]
\definefontfamily[myfonts][rm][nsimsun]
                 [bf=simhei,it=kaiti,bi=simhei]
\setupbodyfont[myfonts,16pt]
\setscript[hanzi]
\stopTEX
\startexample
\startTEXpage[frame=on,offset=4pt,width=8cm]
\setupindenting[first,always,2em]

\title{鲁迅家的后园}

在鲁迅家的后园,可以看见墙外有两株树。一株是
枣树,还有一株也是枣树。\par
这上面的夜的天空,奇怪而高,鲁迅生平没有见过
这样的天空。

\stopTEXpage
\stopexample
\example[option=TEX][zaoshu]{散文示例 1}{\externalfigure[04/zaoshu.pdf][width=.3\textwidth]}

上例未写出作者的名字,以便你能够观察到 \CONTEXT\ 标题之后第一段的首行是不缩进的,这是西文的排版习惯。在使用标题命令前,需要用 \tex{\setupheads}\index[setupheads]{\tex{setupheads}} 为所有标题设定其后第一段的首行必须缩进,见下例。

\startexample
\setupindenting[first,always,2em]
\setupheads[indentnext=yes]

\title{鲁迅家的后园}
% ... 省略正文内容 ...
\stopexample
\example[option=TEX][zaoshu-a]{散文示例 2}{\externalfigure[04/zaoshu-a.pdf][width=.3\textwidth]}

现在可以为文章增加作者信息了,他叫无名氏,见下例。

\startexample
\setupheads[indentnext=yes]
\setupindenting[first,always,2em]

\title{鲁迅家的后园}
\midaligned{无名氏}
% ... 省略正文内容 ...
/BTEX\strut/ETEX
\stopexample
\example[option=TEX][zaoshu-2]{散文示例 2}{\externalfigure[04/zaoshu-2.pdf][width=.3\textwidth]}

上例存在的问题是,作者名字距正文过近,而难以凸显。不要尝试在作者名字之后增加一些空行来解决这个问题。\TeX\ 引擎在遇到多个空行时,它也只是把它们当成一个空行,并将其视为 \type{\par}。在版面的竖直方向,段落之间,或标题与段落之间,或标题与标题之间……增加空白距离,可使用 \type{\blank} 命令\index[blank]{\tex{blank}}。下例在作者和正文之间增加一个空行的距离,只需 \type{\blank[line]};要增加 $n$ 个空行的距离,只需 \type{\blank[n*line]}。

\startexample
\title{鲁迅家的后园}
\midaligned{无名氏}
\blank[line]
% 省略了正文内容
/BTEX\strut/ETEX
\stopexample
\example[option=TEX][zaoshu-3]{散文示例 3}{\externalfigure[04/zaoshu-3.pdf][width=.3\textwidth]}

若需要将标题居中,只需使用 \type{\setuphead}单独为 \type{\title} 设定样式:

\starttyping[option=TEX]
\setuphead[title][align=middle]
\stoptyping

若汉字字族已设定粗体,则可将标题的样式设为粗体,并指定字号级别:

\starttyping[option=TEX]
\setuphead[title][style=\bfc,align=middle]
\stoptyping

例 \in[zaoshu-4] 的排版结果已经基本合规了,只是标题里的汉字的分布有些疏松,原因是汉字之间粘连的伸长特性被激活了,大概是 \CONTEXT\ 过于追求文字居中对齐精度所致。

\startexample
\setupheads[indentnext=yes]
\setuphead[title][style=\bfc,align=middle]
\setupindenting[first,always,2em]

\title{鲁迅家的后园}
\midaligned{无名氏}
% ... 省略了正文内容 ...
/BTEX\strut/ETEX
\stopexample
\example[option=TEX][zaoshu-4]{散文示例 4}{\externalfigure[04/zaoshu-4.pdf][width=.3\textwidth]}

对于上例存在的问题,以前的方案是将 \tex{setuphead} 的参数 \type{align} 的值设定为 \type{{middle,broad}},亦即

\startexample
\setuphead[title][style=\bfc,align={middle,broad}]
\stopexample
\example[option=TEX][zaoshu-5]{散文示例 5}{\externalfigure[04/zaoshu-5.pdf][width=.3\textwidth]}

\noindent 便可适当放松 \CONTEXT\ 过于严格的对齐规则。

不知 \CONTEXT\ 从哪个版本开始,提供了新的参数 \type{center}\index[center]{\type{center} 属性},它与 \type{{middle,broad}} 等效,故而上述设定亦可写为

\startTEX
\setuphead[title][style=\bfc,align=center]
\stopTEX

\noindent 请记住此事,因为以后会经常需要设定其他排版元素的居中对齐,所用参数是相似的,亦即今后在设定某些排版元素的 \type{align} 参数时,至少在中文排版时,建议忘记 \type{middle},只用 \type{center} 即可。

\section[context-world]{正式踏入 \CONTEXT\ 世界}

新手村终究太小了,小到已经不太容易让你尝试越来越多的版命令了。事实上,真正的 \CONTEXT\ 世界用起来要比新手村更为简单,只需用正文环境亦即 \type{text} 环境代替单页环境即可。此外,建议将一切设置排版样式的命令放在正文环境之前,从而在正文环境里,只需要关心文章或书籍的内容。

以下代码应当有助于你看到 \CONTEXT\ 世界大致面目。它是完整的,亦即可将其保存为 \CONTEXT\ 源文件并予以编译。

\starttyping[option=TEX]
% 排版样式
\definefontfamily[myfonts][rm][nsimsun][bf=simhei]
\setupbodyfont[myfonts,10.5pt]
\setscript[hanzi]
\setupheads[indentnext=yes]
\setuphead[title][style=\bfc,align=center]
\setupindenting[first,always,2em]
\setupinterlinespace[line=1.5\bodyfontsize]
% 正文环境
\starttext
\title{鲁迅家的后园}
\midaligned{无名氏}
\blank[line]
在我的后园,可以看见墙外有两株树,一株是枣树,还有一株也是枣树。
... ... ...
\stoptext
\stoptyping

\section{页码}

如果你亲自动手编译了 \in[context-world] 节的 \CONTEXT\ 源文件,应当能看到,排版结果的页眉是有页码的,如图 \in[pagenumber] 所示。这是 \CONTEXT\ 默认的页码样式,即页码出现在每一页,且居中位于页眉,这通常并不合乎多数中文文档的排版习惯,需要设定页码样式。

文章标题所在页面,通常不需要页码,因此需将标题样式将页眉和页脚置空\index[setuphead]{\tex{setuphead}}:

\starttyping[option=TEX]
\setuphead[title][header=empty,footer=empty]
\stoptyping

\placefigure
  [here][pagenumber]
  {\CONTEXT\ 默认页码位置}{\framed{\externalfigure[04/pagenumber.png][width=.8\textwidth]}}

然后,修改页码投放位置,例如将其放在页脚右侧\index[setuppagenumbering]{\tex{setuppagenumbering}}:

\starttyping[option=TEX]
\setuppagenumbering[location={footer,right}]
\stoptyping

\section[style]{内容与样式分离}

用 \CONTEXT\ 或者任何一种 \TEX,保持排版样式与内容的分离永远都是值得鼓励的行为。这种分离极为简单。例如,新建一份文件 foo-env.tex,令其内容为

\startTEX
\definefallbackfamily
    [myfonts][rm][latinmodernroman][range={0x0000-0x0400},force=yes]
\definefontfamily
    [myfonts][rm][nsimsun][bf=simhei,it=kaiti,bi=simhei]
\setupbodyfont[myfonts,16pt]
\setscript[hanzi]

\setupindenting[first,always,2em]
\setupinterlinespace[line=1.5\bodyfontsize]
\setupheads[indentnext=yes]
\setuphead[title][style=\bfc,align=center]
\setuphead[title][header=empty,footer=empty]
\setuppagenumbering[location={footer,right}]
\stopTEX

\noindent foo-env.tex 即为样式文件,它可以重复使用,也可以分享给他人使用。

假设我们在排版一份文档时使用上述 foo-env.tex 文件中的样式,只需用 \tex{environment} 命令\index[environment]{\tex{environment}}将该文件的内容载入即可。例如

\startTEX
\environment foo-env % 文件扩展名可以省略
\starttext
% ... 正文 ...
\stoptext
\stopTEX

样式与内容分离的意义是,当我们在编写文档内容时,不会受到排版样式的任何干扰,甚至完全无需考虑任何与排版有关的事。当内容彻底定稿后,再考虑排版问题。所用的排版样式,可以自行设计,可以复用之前的样式,也可以是从他人那里获得。这是很基本的工程学思想,即在设计上将可以复用的事物与不可复用的事物先隔离开,再以简单的形式构造建立二者的联系。

\subject{结语}

现在,你已经可以用 \CONTEXT\ 写信件、日记、随笔甚至一些读书笔记了。倘若动手尝试了 \type{\chapter} 命令,你甚至能用 \CONTEXT\ 写一本小说,只是风格过于朴素而已。

若想让排版结果更为精致,\CONTEXT\ 博大精深,总有途径能够实现你的想法,前提是你要用心。\TEX\ 之父 Donald Knuth 曾有一言,「我从来也不期盼 {\TEX} 会成为某种万能的排版工具,用于制作一些快速而脏的东西;我只是将其视为一种只要你足够用心就能得到最好结果的东西。」

也许,很多人觉得 \TEX\ 太难了,实际上并非如此。\TEX\ 应该是简单的,而用心……这件事对于大多数人而言,是件难事。
