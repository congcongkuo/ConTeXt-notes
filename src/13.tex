\chapter{写一本书?}

除了之前章节介绍的所有排版元素,一本书还要有目录、引用、脚注、索引、书签等元素。本章对这些元素给出一些示例性质的介绍,以便你能快速将其应用到你的文档里。详细介绍是不太可能的,原因是,即使我愿意为每个元素写出一章的内容详尽介绍,你也未必会看的,一方面是这些元素本身就枯燥无味,另一方面在一般性的场景中,你也无需了解关于它们的太多细节。不过,对每个元素给出简要介绍后,我会附上更为详尽的介绍以备你日后参考之用。

\section{结构}

一本书,通常是由一组文章构成的,可分为序、前言、目录、正文篇章、跋、参考文献、索引、附录等内容,再加上封面,衬页、扉页等页面。对于书籍制作,\CONTEXT\ 对 \type{text} 环境给出了更为细致的划分:

\starttyping[option=TEX]
\starttext
\startfrontmatter
% 此处放置封面、扉页、序、前言、目录等内容
\stopfrontmatter
\startbodymatter
% 此处为正文篇章
\stopbodymatter
\startbackmatter
% 此处方式跋,参考文献列表、索引等内容
\stopbackmatter
\startappendices
% 此处为附录
\stopappendices
\stoptext
\stoptyping

对于上述文档逻辑结构,皆为 \tex{definesectionblock} 命令\cite[sectionblock]定义的区块实例,你也可以用这个命令定义自己的结构。

将一本书的内容全部放在同一份源文件中并不违法,但无论是撰写还是修改必定极为不便,故而通常是构造一份主文件,其内容如下例,而书的各部分内容则存放在一些独立的源文件里,然后用 \type{\input} 命令将其载入到主文件中。

\starttyping[option=TEX]
% 主文件
\starttext
\startfrontmatter
\input cover  % cover.tex:封面
\input preface  % preface.tex:序
\stopfrontmatter
\stoptyping
\starttyping[option=TEX]
\startbodymatter
\input 01  % 01.tex:第一章
\input 02  % 02.tex:第二章
% ... ...
\stopbodymatter
\startbackmatter
\title{参考文献}
\placelistofpublications  % 参考文献列表
\stopbackmatter
\startappendices
\null  % 暂时为空
\stopappendices
\stoptext
\stoptyping

\noindent 主文件 \type{\input} 命令载入的文件通常位于同一目录。

原则上,书籍的所有排版样式皆应在单独的文件中设定,然后使用 \type{\input} 或 \type{\environment} 命令在 \type{text} 环境之前将其载入。例如

\starttyping[option=TEX]
\environment book-style  % 载入book-style.tex
\starttext
% ... ... ...
\stoptext
\stoptyping

\section[TOC]{目录}

\type{\placecontent} 可将全文章节标题及其所在页码等信息汇总为一个列表,以方便读者查阅。例如

\cmdindex{placecontent}
\starttyping[option=TEX]
\usemodule[visual]
\starttext
\title{\fakewords{3}{5}}
\placecontent
\section{\fakewords{3}{5}}
... ... ...
\section{\fakewords{3}{5}}
... ... ...
\stoptext
\stoptyping
\midaligned{%
  \clip[width=\textwidth,height=6cm,voffset=3cm]{%
    \externalfigure[13/toc.pdf][width=\textwidth,frame=on]%
  }%
}

使用 \type{\setupcombinedlist} 可设定目录样式,例如设定可在目录中出现的标题级别以及列表样式。若得到常见的目录样式,只需作以下设定:

\starttyping[option=TEX]
\setupcombinedlist[content][alternative=c]
\stoptyping
\midaligned{%
  \clip[width=\textwidth,height=6cm,voffset=3cm]{%
    \externalfigure[13/toc-2.pdf][width=\textwidth,frame=on]%
  }%
}
\blank
\noindent 目录列表样式参数 \type{alternative} 有 \type{a},\type{b},\type{c},\type{d} 四个值可选,默认是 \type{b}。本文档目录使用的是 \type{d}。

\type{\setupcombinedlist} 亦可用于指定可出现在目录列表中的标题级别,例如

\starttyping[option=TEX]
\setupcombinedlist[content][list={chapter,section}]
\stoptyping

注意,当 \type{\placecontent} 出现在 \type{\chapter} 之后时,生成的目录仅针对该章之内的各节。若是写书,需将 \type{\placecontent} 放在 \type{frontmatter} 环境,例如:

\starttyping[option=TEX]
\startfrontmatter
\title{目录}
\placecontent
\stopfrontmatter
\stoptyping

\noindent 可对全篇被列入目录列表的章节生成目录。

使用 \type{\setuplist} 可对出现在目录列表中相应级别的标题样式分别予以设定,例如

\starttyping[option=TEX]
\setuplist[chapter]
          [alternative=a,
           before={\blank[halfline]},after={\blank[halfline]},style=bold]
\setuplist[section]
          [alternative=d,style=normal,pagestyle=smallbold]
\stoptyping

\CONTEXT\ 默认不允许无编号标题出现在目录中,但是倘若对无编号标题,例如 \type{\title} 作以下设定

\starttyping[option=TEX]
\setuphead[title][incrementnumber=list]
\stoptyping

\noindent 之后便可将 \type{\title} 添加到目录列表,即

\starttyping[option=TEX]
\setupcombinedlist[content][list={title,chapter,section}]
\stoptyping

需要注意的是,在 \type{frontmatter} 环境中放置目录列表时,若使用以下代码

\starttyping[option=TEX]
\startfrontmatter
\title{目录}
\placecontent
\stopfrontmatter
\stoptyping

\noindent 由于此时 \type{\title} 已被列入目录列表,因此 \type{\title{目录}} 本身会出现在目录列表中。为避免这一问题,需要为目录页单独定义一个标题。\CONTEXT\ 支持我们定义自己的标题,例如

\starttyping[option=TEX]
\definehead[TOC][title]
\stoptyping

\noindent 定义了一个新的标题 \type{\TOC},它与 \type{\title} 的样式相同。在 \type{frontmatter} 环境中使用 \type{\TOC}:

\starttyping[option=TEX]
\startfrontmatter
\TOC{目录}
\placecontent
\stopfrontmatter
\stoptyping

\noindent 由于我们并未将 \type{\TOC} 列入目录列表,因此上述问题得以解决。

文档\cite[toc]对目录给出了更为全面的介绍,可作进一步的参考。

\section[reference]{引用}

在之前的列表、表格、插图和数学公式等章节中,已简略介绍了 \CONTEXT\ 引用的用法。\CONTEXT\ 的标题也支持引用。例如,本节的标题对应的排版命令是

\starttyping[option=TEX]
\section[reference]{引用}
\stoptyping

可以在文章几乎任何一个位置,像下面这样引用本节:

\startexample
我在 \at[reference] 页 \in[reference] 节\about[reference]中的一些内容。
\stopexample
\typeexample[optio=TEX]
\startframedtext[width=broad]
\getexample
\stopframedtext

使用 \type{\textreference} 可在文档几乎任何位置插入引用。例如

\cmdindex{textreference}
\startexample
我在此处放置了一个引用\textreference[myref]{一个引用}。
\stopexample
\typeexample[option=TEX]
\startframedtext[width=broad]
\getexample
\stopframedtext

\startexample
我在此处使用一个引用,它是 \at[myref] 的「\in[myref]」。
\stopexample
\typeexample[option=TEX]
\startframedtext[width=broad]
\getexample
\stopframedtext

\section{脚注}

若需要对书籍内容中一些文字进行注解,可采用脚注形式。\tex{footnote} 命令可在待注解的文字之后插入脚注,并自动生成序号。例如

\startexample
我以脚注的形式告诉你一个关于 \CONTEXT\ 脚注的秘密\footnote{\CONTEXT\ 的脚注默认不支持中文断行,详见 \CONTEXT\ 邮件列表上的一次讨论\cite[chinese-footnotes]。}。
\stopexample
\typeexample[option=TEX]
\startframedtext[width=broad]
\getexample
\stopframedtext

解决脚注中较长的中文无法断行这一问题,一种方法是在脚注中使用 \type{\setscript[hanzi]} 强行开启中文断行功能,为了简便,可以专门定义一个宏替代 \type{\footnote}:

\starttyping[option=TEX]
\def\zhfootnote#1{\setscript[hanzi]#1}
\stoptyping

\noindent 也可以用 \CONTEXT\ 超人 Wolfgang Schuster 提供的方法,使用 \type{setups} 机制\cite[setups],为脚注构造专用的设定,见下例。

\starttyping[option=TEX]
\startsetups footnote:hanzi
\setscript[hanzi]
\stopsetups
\setupnote[footnote][setups={footnote:hanzi}]
\stoptyping

\section{索引}

索引通常放在 \type{backmatter} 环境,即附在书的正文之后,以便检索在正文某页检索一些关键词。这些关键词在正文中需由 \type{\index} 给出。例如

\starttyping[option=TEX]
\startbackmatter
\title{索引}
\placeindex
\stopbackmatter
\stoptyping

\startexample
我在此演示 \tex{placeindex}\index[placeindex]{\tex{placeindex}}
和 \type{\index}\index[index]{\type{\index}} 的用法。
\stopexample
\typeexample[option=TEX]
\startframedtext[width=broad]
\getexample
\stopframedtext

\noindent \tex{placeindex} 产生的结果见本文档附录部分的「索引」。

\section{书签}

与目录类似,对于内容较多的 PDF 文档,提供书签(Bookmark)亦可便于他人阅读。书签通常显示于 PDF 阅读器的侧栏,如图 \in[13-bookmarks] 所示,点击某个书签便可跳转至其关联的页面。

\placefigure[here][13-bookmarks]{PDF 书签}{\externalfigure[13/bookmarks.png][width=.7\textwidth]}

为 PDF 文件制作书签,只需在样式文件中添加以下语句:

\starttyping[option=TEX]
\setupinteraction[state=start,focus=standard]
\setupinteractionscreen[option=bookmark]
\placebookmarks[title,chapter,section][title,chapter]
\stoptyping

\noindent 其中,\type{\setupinteraction} 用于开启 PDF 的用户交互特性。\type{\setupinteractionscreen} 用于设定 PDF 文件被阅读器打开后,以何种形式如何呈现在屏幕上,若其 \type{option} 值为 \type{bookmark},则文件打开后,会自动开启阅读器的侧边栏并显示书签;若设置 \type{option} 为 \type{max},则文件在被打开后会全屏显示。上述 \type{\placebookmarks} 语句的用途是设置可出现在书签栏的标题级别,且仅允许 \type{\title} 和 \type{\chapter} 级别的标题,其子标题列表可被展开。

需要注意的是,\CONTEXT\ 同样默认无编号标题不被列入书签,但是倘若做以下设定

\starttyping[option=TEX]
\setuphead[title][incrementnumber=list]
\stoptyping

\noindent 则 \type{\title} 亦可出现在书签列表中。

还需要注意一点, 书签功能取决于你所用的 PDF 阅读器是否支持。此外,你的 PDF 阅读器可能会将 \ConTeXt\ 生成的书签视为索引(Index),而其本身则提供了另一个叫作书签的功能,允许用户手动在侧边栏为文档的某一页建立链接,与 \CONTEXT\ 的书签原理相同。

\section{双面}

为了省纸,书籍通常是双面排版印刷,此时页码的位置要对开设定。如果订口和翻口是非对称的,还需要按双面模式设定页面布局。通常是翻口和底部(地脚)的留白要大一些,以便人们在读书时写注解。经典的双面排印布局是 Van de Graaf 的 Conon 版式,用 \CONTEXT\ 的页面布局可表达为

\startTEX
\usemodule[visual]
% 纸张为 A4
\setuppapersize[A4]
% Van de Graaf Conon 版式
\setuplayout
        [backspace=.111\paperwidth,
         width=.667\paperwidth,
         topspace=.056\paperheight,
         height=.778\paperheight]
% 开启双面对开模式
\setuppagenumbering[alternative=doublesided]
\starttext
\dorecurse{30}{\fakewords{50}{150}\par}
\stoptext
\stopTEX

\placefigure[force][conon]{Van deGraaf Conon 版式}{\externalfigure[13/van.png][width=.6\textwidth]}

\section{页眉}

在页眉区域放置内容,需要用 \tex{setupheadertexts} 命令\cmdindex{setupheadertexts}。在单面排印的情况下,该命令支持三个参数,第一个参数是可选的,若不设定,则值默认是 \type{text},表示版心上下的页眉和页脚局部区域,也可以将其值设为 \type{margin},此时表示的是订口和翻口上下的页眉和页脚局部区域。

下例设定了正文区域上方的页眉左侧和右侧区域的内容。

\startTEX
\setupheadertexts[页眉左侧][页眉右侧]
\stopTEX

如果你需要在订口和翻口上方的页眉区域放置内容,可按下例实现。

\startTEX
\setupheadertexts[margin][订口上方的页眉][翻口上方的页眉]
\stopTEX

在双面对开模式下,页眉和页脚的设定则需要五个参数,第一个参数依然是 \type{text} 或 type{margin},至于后四个参数,其中前两个用于设定偶数页的页眉和页脚的左右两侧,后两个则用于设定奇数页的页眉左右两侧。下例是对偶数页和奇数页的正文上方页眉内容的设定。

\startTEX
\setupheadertexts[偶数页的页眉左侧][偶数页的页眉右侧]
                 [奇数页的页眉左侧][奇数页的页眉右侧]
\stopTEX

下例是偶数页和奇数页的订口和翻口上方的页眉内容的设定。

\startTEX
\setupheadertexts
  [margin]
  [偶数页的页眉左侧][偶数页的页眉右侧][奇数页的页眉左侧][奇数页的页眉右侧]
\stopTEX

页眉区域通常可以放置章节标题以及页码等内容。本文档主要作为屏幕读物,而非印刷读物,故而采用了单面模式,但是页眉是按双面对开模式设定的,如下:

\startTEX
\def\CurrentChapter{%
  第 \headnumber[chapter]\ 章\kern 1em\getmarking[chapter]%
}
\def\CurrentSection{%
  \headnumber[section][current]%
  \kern 2em%
  \getmarking[section][current]%
}
\setupheadertexts[\CurrentChapter][pagenumber][pagenumber][\CurrentSection]
\stopTEX

\noindent 在单面模式下,偶数页的页眉左侧和右侧显示的分别是当前小节的标题和页码,而奇数页的页眉左侧和右侧显示的分别是当前章的标题和页码,并未按上述设定。但是若开启了双面对开模式,奇数页的页眉便会按上述设定,页码在左侧,章的标题在右侧。

本文档的页眉设定用了几个之前未曾提及的命令,其中 \tex{headnumber}\cmdindex{headnumber} 可以取得当前章节的序号,\tex{getmarking}\cmdindex{getmarking} 可以取得当前章节的标题,\tex{kern}\cmdindex{kern} 命令用于制作指定宽度的间距。此外,我还需要再次提醒,要注意上述宏定义里的注释符,它们的用途是消除其后的换行符,以防在宏展开时这些换行符被当成空格而污染排版内容。

也许你忽然想到,如果我想在正文上方的页眉区域或下方的页脚区域的中部放置内容,该如何实现呢?页眉和页脚的设定命令皆支持单参数形式,可用于在正文上方页眉或下方页脚中部放置内容,下例绘制了页眉底线,并居中放置了一些内容。

\startTEX
\setupheadertexts
  [\framed[width=\textwidth, frame=off, bottomframe=on]{页眉内容}]
\stopTEX

你也能将上例扩展为在页眉的左侧、中部和右侧放置内容,同时带有底线的效果,而无需使用 \tex{setupheadertexts} 命令的多参数形式,只是在双面对开模式下,你需要用 \tex{ifodd} 命令\cmdindex{ifodd}判断当前页码的奇偶性,然后相应调整左右侧内容,下例可作为参考。

\startTEX
\defineframed[framedheader]
             [width=\textwidth, frame=off, bottomframe=on]
\def\myheader{%
  \ifodd\pagenumber
    \framedheader{左侧\hfill 中部\hfill\pagenumber}%
  \else
    \framedheader{\pagenumber\hfill 中部\hfill 右侧}%
  \fi%
}
\setupheadertexts[\myheader]
\stopTEX
\placefigure[force][myheader]{自定义页眉}{\externalfigure[13/myheader.png][width=\textwidth]}

\section{页脚}

在 \in[footer] 节已经用 \tex{setupfootertexts} 命令\cmdindex{setupfootertexts}为幻灯片设置了页脚区域,该命令的参数形式与 \tex{setupheadertexts} 相同,故其用法不再赘述。

\subject{结语}

\TEX\ 系统复杂且难以掌握,主要原因在于排版本身便是极为复杂的事情,页眉区区方寸之地,若想随心所欲设置一些内容,已是颇为不易了,而写书更是工程级别的任务。绝大多数人,一生之中,工作量最大的单个作品大概是自己的学位论文,本科一二十页,硕士四五十页;到了博士阶段,学位论文有百余页,算得上书的级别。换言之,\TEX\ 在写博士论文这种程度的文档时,方能刚刚开始全面展现其功能和优势。换言之,用 \TEX\ 排版一份文档的难度取决于该文档要写什么以及如何写,不可脱离文档自身的内容和形式去看待某个排版系统的优点和缺点。
