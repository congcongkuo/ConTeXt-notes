\chapter[Installation]{命令行}

无论是安装还是使用 \CONTEXT,皆需要对命令行环境有所了解。原本未有介绍这方面知识的计划,但是考虑到我正在写一份世上最好的 \CONTEXT\ 入门文档,便有了些许动力。本章先分别介绍 Windows、Linux 和 macOS 系统的命令行环境的基本用法,以刚好满足安装和运行 \CONTEXT\ 的需求为要,然后再分别介绍在这三个系统里如何用命令行的方式安装 \CONTEXT\ 环境。

若你对命令行环境已颇为熟悉,可直接阅读从 \in[installation] 开始阅读,否则要有耐心从一个简单任务开始熟悉它的基本用法。这个任务是,在命令行环境里,创建 foo 目录,并在该目录内创建一份 Shell 脚本,令其可在命令行窗口中输出\boxquote{\type{不怕命令行}}。

\section{Windows 命令行}

Windows 用户似乎天生畏惧甚至厌憎命令行环境,甚至很多人认为它早已被淘汰的上个世纪的东西。实际上这种认识是错误的,原因很简单,最新的 Windows 版本依然不仅不敢不提供它,反而处心积虑强化它。

在 Windows 系统中打开历史最为悠久的那个命令行窗口,有很多种方法,其中最快的应当是使用如图 \in[win-r] 所示的组合键「Win + R」,打开「运行」对话框,在其中输入\boxquote{\type{cmd}},然后点击对话框上的「确定」按钮或单击回车键(Enter 键),便可打开与图 \in[cmd-window] 类似的命令行窗口,只不过你看到的应该是背景为黑色的窗口——我为了打印这一页时能省一些墨,将其改成了白色。

\startplacefigure[location=none]
\startfloatcombination[nx=2,ny=1]
\startplacefigure[title={Win + R 组合键},reference=win-r]
\externalfigure[01/win-r.jpg][height=5.45cm]
\stopplacefigure
\startplacefigure[title={Windows 命令行窗口},reference=cmd-window]
\externalfigure[01/cmd-window.png][height=5.45cm]
\stopplacefigure
\stopfloatcombination
\stopplacefigure

在命令行窗口里,所有要执行的命令皆在\boxquote{\type{>}}符号后输入,故而该符号叫作命令提示符。{\bf 需要注意,在输入命令之前,请查看输入法状态,将其切换为英文输入状态。}

任何一条命令都是在某个目录下执行的,该目录称为工作目录。命令提示符左侧内容表示一个文件目录,它就是工作目录。在命令行窗口中输入命令\boxquote{\type{d:}},然后单击回车键便可执行该命令,结果是工作目录被切换为 d 盘。在命令行环境里,输入命令后,必须单击回车键键,方能使命令得以执行。

现在,尝试执行命令\boxquote{\type{md foo}}。该命令可在 d 盘创建目录 foo,然后执行命令\boxquote{\type{cd foo}},将工作目录切换为 foo。现在你的工作目录就是 \type{d:\foo} 或 \type{D:\foo}——Windows 的命令行语句里,字母不区分大小写。

形如 \type{d:\foo} 这样的形式称为绝对路径,反斜线 \type{\} 称为路径分隔符。有相对路径吗?有。在 \type{d:\foo} 中执行命令\boxquote{\type{cd ..}},便可将工作目录切换到上一级,即 d 盘。符号\boxquote{\type{..}}便是相对路径,表示工作目录的上级目录。在 \type{d:\foo} 中,若执行命令\boxquote{\type{cd ..\foo}},则工作目录不会发生变化,因为 \type{d:\foo} 的上级目录的子目录 foo 依然是 \type{d:\foo}。形如 \type{..\foo} 的路径便是常见的相对路径形式。

在 d 盘执行命令\boxquote{\type{cd ..}},工作目录会发变化吗?不会。因为 Windows 任何一个盘符,它的上一级目录为空。

理解上述内容,你应该能够凭借几个简单的命令便可遨游 Windows 文件系统了,只是我们的任务尚未完成。现在可将该任务具体为,在 \type{d:\foo} 中创建一份 Shell 脚本,执行该脚本,在命令行窗口中输出「\type{不怕命令行}」。在 Windows 系统中,Shell 脚本即命令行批处理文件,其扩展名通常为 \type{.bat}。在实现该脚本之前,需要了解 \type{echo} 命令的用法。

\type{echo} 命令亦称回显命令,其功能是读取一些文字,然后将其原样输出。例如,

\startmycmd
> echo 不怕命令行 
不怕命令行
\stopmycmd

\noindent 上述代码的第一行是执行\boxquote{\type{echo}} 命令,第二行是命令的输出结果。

似乎 \type{echo} 是一个什么都不会做的命令,这样的命令有什么用呢?它的一个用途是,通过命令行的输出重定向机制,将一些内容写入文件。例如,

\startmycmd
d:\foo> echo @echo off > foo.bat
\stopmycmd

\noindent 上述命令通过输出重定向符\boxquote{\type{>}}将原本会输出到命令行窗口的文字\boxquote{\type{@echo off}}输出到文件 \type{d:\foo\foo.bat} 。倘若该文件事先并不存在,系统会自动创建它。

现在,用 \type{echo} 命令向文件 \type{d:\foo\foo.bat} 的尾部追加一行文字:

\startmycmd
> echo echo 不怕命令行 >> foo.bat
\stopmycmd

\noindent 注意,向一份已存在的文件追加内容,所用的输出重定向符号是\boxquote{\type{>>}},在这种情况下若使用\boxquote{\type{>}},则文件原有内容会被追加的内容完全替换。

现在,略有纪念意义的时刻到了,这可能是你此生首个甚至也可能是最后一个批处理文件,执行它吧!

\startmycmd
> .\foo.bat
不怕命令行
\stopmycmd

\noindent 注意,上述命令里的\boxquote{\type{.}}表示工作目录本身,故而\boxquote{\type{.\foo.bat}}也是一种相对路径,表示工作目录里的 foo.bat 文件。不过,在 Windows 命令行环境里,这种形式的相对路径可以忽略,亦即上述命令可写为

\startmycmd
> foo.bat
不怕命令行
\stopmycmd

\noindent 在执行某个程序或批处理文件时,倘若未给出其路径,Windows 系统默认先从工作目录中搜索文件,若未搜到,才会在系统环境变量 \type{PATH} 设定的路径中搜索。

系统环境变量 \type{PATH} 是什么呢?既然是变量,必定有值,其值是绝对路径集,以下命令可以查看:

\startmycmd
> echo %PATH%
\stopmycmd

\noindent 顺便指出,这是 \type{echo} 命令的另一种用途,即查看环境变量的值。

使用 \type{setx} 命令可以将上述示例创建的批处理文件 foo.bat 所在目录 \type{d:\foo} 追加至 \type{PATH} 变量现有路径集的尾部,如下:

\startmycmd
> setx /M PATH "%PATH;%d:\foo"
\stopmycmd

\noindent 务必注意,该命令仅在「以管理员身份」启动的命令行窗口中生效。在 Windows 开始菜单里的搜索栏,输入\boxquote{\type{cmd}}并单击回车键键提交,然后鼠标右键单击搜索结果,在弹出的菜单中选择「以管理员身份运行」。

验证 \type{d:\foo} 是否被成功添加到系统 \type{PATH} 变量,只需在除 \type{d:\foo} 之外的任一目录验证能否执行 foo.bat,例如

\startmycmd
> c:
> cd windows\system32  
> foo.bat
不怕命令行  
\stopmycmd

\useURL[Windows 命令行][https://www.bilibili.com/video/BV1vk4y1h7LE/]
若不知如何以管理员身份运行命令行窗口,亦可通过图形界面设置 \type{PATH} 变量。我已将上述构建 \type{d:\foo\foo.bat} 以及如何通过图形界面设置系统 \type{PATH} 变量等过程录制为视频,网络链接为\boxquote{\from[Windows 命令行]},从而避免层峦叠障的 Windows 窗口截图占据本章太多篇幅。

\section{Linux 终端}

在 Linux 系统中,命令行窗口通常称作「终端(Terminal)」。对于 Linux 用户而言,我可以放心假设他们已经知道如何开启终端窗口,甚至我也能假设他们已经很熟悉下文所讲的一切了,这就是 Linux 世界的风采。

现在我们进入 \type{$HOME} 目录,亦即 \type{~} 目录,在其下创建子目录 foo,如下

\startmycmd
$ cd ~
$ mkdir foo
\stopmycmd

\noindent Linux 的命令提示符通常是 \type{$}。接下来进入 foo 目录,用命令输出重定向,将 \type{echo} 命令的输出结果写入 foo.sh 文件:

\startmycmd
$ cd foo
$ echo #!/bin/bash > foo.sh
$ echo echo 不怕命令行 >> foo.sh
\stopmycmd

\noindent 使用 \type{chmod} 命令为 foo.sh 增加可执行权限,让它像程序一样运行:

\startmycmd
$ chmod +x foo.sh
\stopmycmd

\noindent 运行 foo.sh:

\startmycmd
$ ./foo.sh
不怕命令行
\stopmycmd

以下命令将 \type{~/foo} 添加至系统环境变量 \type{PATH} 并使之生效,

\startmycmd
$ cd ~
$ echo 'export PATH=~/foo:$PATH' >> .bashrc
$ source .bashrc
\stopmycmd

注意,上述代码中的单引号是必须的,它能抑制所包含的 Bash 变量的展开,即抑制 \type{$PATH} 的展开。此外,等号两侧不能出现空格。

在任一目录下执行 foo.sh 以验证 \type{~/foo} 是否已被添加至 \type{PATH} 变量,例如

\startmycmd
$ cd /tmp
$ foo.sh
不怕命令行
\stopmycmd

若想对 Bash Shell 有更多的了解,可阅读拙作:

\startitemize[packed]
\item 听说你讨厌 Bash\cite[bash-haters]
\item 写给高年级小学生的 Bash 指南\cite[bash-tutor]
\item Bash 之丑陋\cite[ugly-bash]。
\stopitemize
  
\section{macOS 终端}

macOS 和 Linux 皆为 Unix-like(类 Unix)系统,故而二者终端环境的用法近乎相同,如果你知道如何开启 macOS 的终端窗口,则上文所述的 Linux 命令的用法也适于 macOS。不过,macOS 从 Catalina 版开始,默认的 Shell 不再是 Bash,而是 zsh,故而在设置 \type{PATH} 变量时,命令需变更为

\language[en]
\startmycmd
$ cd ~
$ echo 'export PATH=~/foo:$PATH' >> .zshrc
$ source .zshrc
\stopmycmd

\section[installation]{为 Windows 安装 \CONTEXT}

\useURL[installation][https://wiki.contextgarden.net/Introduction/Installation]
\CONTEXT\ 的最新版本是 \CONTEXT\ LMTX,面向 Windows 系统的安装包可从 \boxquote{\from[installation]} 获取,例如 Intel 或 AMD 的 64 位 CPU 架构的 Windows 机器,选择下载 X86 64bits 版本即可。下文以该版本的安装包为例,讲述 \CONTEXT\ LMTX 的安装过程。

\useURL[win-64-version][https://lmtx.pragma-ade.com/install-lmtx/context-win64.zip]
首先,从 \boxquote{\from[win-64-version]} 下载面向 Windows 64 位系统的安装包 \type{context-win64.zip},假设在 \type{d:\} 将其解开,得目录 \type{context-win64},其结构当如图 \in[win64] 所示。完成解包工作,context-win64.zip 包就没用了,可以删除。

然后,将 \type{context-win64} 目录改名为 \type{context},继而打开命令行窗口,依序执行以下命令:

\startmycmd
> d:
> cd context
> install.bat
> setpath.bat
> mtxrun --generate
\stopmycmd

\placefigure[here][win64]{\CONTEXT\ LMTX 的 windows 64 位安装包结构}{\externalfigure[01/context-win64.pdf][width=.8\textwidth]}

上述命令里,批处理文件 install.bat 可从网络上下载 \CONTEXT\ LMTX 的所有文件,存放在工作目录。安装时长取决于网络下载速度。\CONTEXT\ LMTX 的服务器在境外,由于国内众所周知的原因,导致文件下载速度可能会非常缓慢,而且安装过程甚至可能中断,需要你多次执行 install.bat 文件,方能完成整个安装过程。幸好 install.bat 是增量安装,即使有所中断,每次它会从中断处开始安装,而非从头重新安装。后文会给出符合国情的快捷方案。

从现在开始,称 \type{d:\context} 目录为 {\bf\CONTEXT\ 安装目录}。若想验证 \CONTEXT\ LMTX 是否安装完全,可在 \CONTEXT\ 安装目录执行以下命令:

\startmycmd
> setpath.bat
> md test
> cd test
> echo \startTEXpage[frame=on,offset=1pt] > foo.tex
> echo Hello \CONTEXT! >> foo.tex
> echo \stopTEXpage >> foo.tex
> context foo.tex
\stopmycmd

\noindent 倘若在 \type{d:\context\test} 目录下能够得到 \type{foo.pdf} 文件,且其内容为 \lower.3em\hbox{\externalfigure[01/foo.pdf]},则意味着已成功安装 \CONTEXT\ LMTX。

最后一步,将 \type{d:\context\tex\texmf-win64\bin} 添加到系统 \type{PATH} 变量,便可在任一目录使用 \type{context} 命令将扩展名为 \type{.tex} 的文本文件编译为同名的 PDF 文件。

\useURL[sumatra][https://www.sumatrapdfreader.org/free-pdf-reader]
上述命令创建的 foo.tex 文件,你可以用任何一款文本编辑器打开,作一些编辑,例如用 Windows 的记事本程序打开它:

\startmycmd
> notepad foo.tex
\stopmycmd

如果你没有 PDF 阅读器,或者每次重新编译 \type{.tex} 文件时,你的 PDF 阅读器没有自动更新显示 PDF 文件的内容,我建议你用 Sumatra PDF 阅读器,其下载页面位于\boxquote{\from[sumatra]}。

\section{为 Linux 安装 \CONTEXT}

假设你的 Linux 是 Inter 或 AMD 架构的 64 位系统,并且你希望将 \CONTEXT\ LMTX 安装在 \type{~/opt} 目录,亦即 \type{~/opt},则整个安装过程可表述为以下 Bash 命令:

\startmycmd
$ mkdir -p ~/opt/context
$ cd ~/opt/context
$ wget https://lmtx.pragma-ade.com/install-lmtx/context-linux-64.zip
$ unzip context-linux-64.zip
$ sh install.sh
$ echo 'export PATH=~/opt/context/tex/texmf-linux-64/bin:$PATH' >> ~/.bashrc
$ mtxrun --generate
$ rm context-linux-64.zip
\stopmycmd

\section{为 macOS 安装 \CONTEXT}

在 macOS 系统中安装 \CONTEXT\ LMTX 所用命令与上述 Linux 系统安装命令近乎相同,只是 macOS 环境里默认可能没有 \type{wget} 命令,不过你可以用 \type{curl} 命令替代它。以下命令可获得面向 macOS 64 位系统的 \CONTEXT\ LMTX 安装包:

\startmycmd
$ curl -O https://lmtx.pragma-ade.com/install-lmtx/context-osx-64.zip
\stopmycmd

在安装过程结束后,你可能需要消除一些文件的隔离标记。假设你的 \CONTEXT\ 安装目录为 \type{~/opt/context},可执行以下命令完成该工作。

\startmycmd 
$ sudo xattr -r -d com.apple.quarantine ~/opt/context/bin/mtxrun
$ sudo xattr -r -d com.apple.quarantine ~/opt/context/tex/texmf-osx-64/bin/*
\stopmycmd

\section{接力棒}

\useURL[installation][https://wiki.contextgarden.net/Introduction/Installation]
上述安装方法源于\boxquote{\switchtobodyfont[9pt]\from[installation]},其中面向 Windows 和 Linux 的安装过程,我皆已实践,除文件下载速度太慢以及安装过程偶有中断之外,没有其他问题。面向 macOS 系统的安装过程,我没有条件予以验证,只能纸上谈兵。

为了加快安装进程,我将 Windows 和 Linux 里装好的 \CONTEXT\ LMTX 分别打包,以时间为后缀,上传到了网盘。你可以下载它们,只需将其解包到指定目录,便可拥有一个版本略微落后的 \CONTEXT\ LMTX 环境。之后只需执行安装目录里的 install 脚本便可追及最新版本——相当于意外中断后继续安装。由于这部分内容经常变化,而本文档更新周期通常以年为单位,故而我将其做成一个网页:

\useURL[my-shoulder][https://zhuanlan.zhihu.com/p/1943204598711054617]
\midaligned{\boxquote{\from[my-shoulder]}}。

\noindent 只需按照该网页的说明,便可创建一个纯净的 \CONTEXT\ LMTX 环境。

\section[ctx-in-texlive]{\TEX\ Live 中的 \CONTEXT}

\TEX\ Live 是迄今为止最为权威的 \TeX\ 系统,\LaTeX\ 用户大多使用该系统,不过该系统也收录了 \CONTEXT\ 包。相比于 \CONTEXT,\TEX\ Live 是更为宏大的 \TeX\ 体系,前者可谓是后者的一颗行星。由于 \TEX\ Live 提供了图形界面的安装程序,如果你始终都不得命令行其门而入,可以考虑用该程序安装 \CONTEXT\ LMTX。

鉴于本章的主题是在命令行环境里安装 \CONTEXT,倘若在此讲述 \TEX\ Live 系统的图形化安装过程,结果必定会喧宾夺主,仅是安装过程的一些截图所占的篇幅就已经超过了前文所有内容。不过,为了让你能多一条路,该安装方式在上一节的 \CONTEXT\ 接力棒的在线文档中里也有记述,并且以视频的形式演示了安装过程。

\subject{结语}

本章内容,其意不止在于讲述如何在命令行环境里安装 \CONTEXT\ 包,它更希望你能熟悉甚至习惯以命令行的方式工作。图形界面程序虽然让你觉得更友好,但永远也不要觉得命令行很原始落后。实际上在大多数生产场景里,命令行环境是火焰喷射器,而图形界面程序像是一根又一根精美的火柴,上面还能雕梁画栋。

即使 Windows 的命令行批处理文件在功能上远弱于 Linux 和 macOS 里的 Shell 脚本,当你在对其有所掌握时,依然能从中受益良多,因为这类脚本通常可以将许多繁琐的工作变为自动化的过程。

你对命令行环境的重视,实质上也能体现对 \TeX\ 开发者们的的尊重。大多数 \TeX\ 开发者,是在自己的爱好驱动下工作的,而非产品经理的爱好。若你作为用户,熟悉命令行环境的用法,便意味着这些开发者可以从编写图形界面的繁重工作中解放出来,将生命更多地燃烧在 \TeX\ 功能的开发上。实际上于我而言,也能节省很多精力,我并不想用大量的截图和文字记述任何一个软件的安装过程。
