\chapter[Installation]{不怕命令行 }

无论是安装还是使用 \ConTeXt,皆需要对命令行环境有所了解。原本未有介绍这方面知识的计划,但是考虑到我正在写一份世上最好的 \ConTeXt\ 入门文档,便有了些许动力。本章先分别介绍 Windows、Linux 和 macOS 系统的命令行环境的基本用法,以刚好满足安装和运行 \ConTeXt\ 的需求为要。倘若对命令行环境已颇为熟悉,可直接阅读 \in[installation] 和 \in[ctx-in-texlive] 节。

\section{任务}

使用命令行环境,在文件系统中,创建一个名为 foo 的目录,在该目录内创建一份 Shell 脚本,令其可在命令行窗口中输出「不怕命令行」,执行该脚本,查看其输出。

\section{Windows 命令行}

Windows 用户似乎畏惧甚至厌憎命令行环境,甚至很多人认为命令行环境是早已被淘汰的上个世纪的产物,因此要教会他们如何使用命令行环境,通常会有些麻烦,我当勉力为之。

在 Windows 系统中打开一个命令行窗口,有很多种方法,其中最快的应当是使用如图 \in[win-r] 所示快捷键「Win + R」,打开「运行」对话框,在其中输入「cmd」后点击「确定」按钮或单击「Enter」键,即可打开与图 \in[cmd-window] 类似的命令行窗口。

\startplacefigure[location=none]
\startfloatcombination[nx=2,ny=1]
\startplacefigure[title={「\type{Win + R}」 组合键},reference=win-r]
\externalfigure[01/win-r.jpg][height=6cm]
\stopplacefigure
\startplacefigure[title={Windows 命令行窗口},reference=cmd-window]
\externalfigure[01/cmd-window.png][height=6cm]
\stopplacefigure
\stopfloatcombination
\stopplacefigure

所有要执行的命令皆是在「\type{>}」符号后输入,因而该符号名曰「命令提示符」。命令提示符左侧的内容是一个文件目录,名曰「当前目录」或「工作目录」。在命令行环境里,任何命令皆是在某个目录中执行的。在开始尝试输入命令之前,请将输入法切换为英文输入状态。

在命令行窗口中输入命令「d:」,然后单击「Enter」键执行该命令,可将当前目录切换为 D 盘。在命令行环境里,输入命令后,必须单击「Enter」键,方能使命令得以执行。

执行命令「\type{md foo}」,可在 D 盘根目录创建目录 foo,然后执行命令「\type{cd foo}」,将当前目录切换为 foo,即 \type{D:\foo}。

诸如 \type{D:\} 和 \type{D:\foo} 这样的表示形式统称为路径,更为准确地说,是绝对路径。有相对路径吗?有。在 \type{D:\foo} 中执行命令「\type{cd ..}」,便可将当前目录返回上一级,即 \type{D:\}。在 \type{D:\foo} 中执行命令「\type{cd ..\foo}」,则当前目录不会发生变化,因为 \type{D:\foo} 的上一级目录的子目录 foo 依然是 \type{D:\foo} 。形如 \type{..\foo} 这样的路径便是相对路径。

在当前目录 \type{D:\} 中执行命令「\type{cd ..}」,当前目录会发生什么变化?不会发生任何变化,因为 \type{D:\} 是 D 盘的最顶层目录,亦即它的上一级目录为空。

掌握了上述命令,便可在 Windows 文件系统中畅游无阻了,然而我们的任务尚未完成。在 Windows 系统中,该任务可描述为,在 \type{D:\foo} 中创建一份 Shell 脚本,执行该脚本,在命令行窗口中输出「不怕命令行」。在 Windows 系统中,Shell 脚本即批处理文件——扩展名为「\type{.bat}」的纯文本文件。在制作这份批处理文件之前,需要了解「\type{echo}」命令的基本用法。

「\type{echo}」命令,如同我们对着幽深的山谷呼喊而产生回声的过程,它读取一段文本,然后在命令行窗口中原样输出。例如,

\starttyping
D:\foo> echo 不怕命令行 
不怕命令行
\stoptyping

似乎「\type{echo}」命令是一个什么都不会做的命令,这样的命令有什么用呢?它有一个用途是,利用输出重定向,将一些内容写入文本文件。例如,

\starttyping
D:\foo> echo @echo off > foo.bat
\stoptyping

\noindent 上述命令通过命令输出重定向符「\type{>}」将「\type{echo}」原本会输出到命令行窗口的文字「@echo off」重定向为输出到文件 \type{D:\foo\foo.bat} 。倘若该文件事先并不存在,上述命令会自动创建它。

现在,使用「\type{echo}」命令向文件 \type{D:\foo\foo.bat} 增加一行文字:

\starttyping
D:\foo> echo echo 不怕命令行 >> foo.bat
\stoptyping

\noindent 注意,向指定文件追加内容,需要使用「\type{>>}」,倘若使用「\type{>}」,则文件原有内容会被全部替换。

现在,略有纪念意义的时刻到了,你可能已经创建了人生中第一份 Windows 批处理文件,执行它吧!

\starttyping
D:\foo> .\foo.bat
不怕命令行
\stoptyping

\noindent 注意,上述命令使用了相对路径的第二种形式,路径中的「\type{.}」表示当前目录。事实上,在 Windows 命令行环境里,这种形式的相对路径可以忽略,亦即上述命令可以改为

\starttyping
D:\foo> foo.bat
不怕命令行
\stoptyping

\noindent 在执行某个程序或批处理文件时,倘若未给出其路径,Windows 系统默认先从当前目录中搜索文件,若未搜到,才会在系统环境变量 \type{PATH} 设定的路径中搜索。

系统环境变量 \type{PATH} 是什么呢?既然是变量,必定有值,其值是绝对路径集,执行以下命令可以查看:

\starttyping
D:\foo> echo %PATH%
\stoptyping

\noindent 顺便指出,这是「\type{echo}」命令的另一种用途。

使用以下「\type{setx}」命令可以将上述示例创建的批处理文件 foo.bat 所在目录 \type{D:\foo} 追加至 \type{PATH} 变量现有路径集的尾部:

\starttyping
D:\foo> setx /M PATH "%PATH%d:\foo;"
\stoptyping

\noindent 务必注意,该命令仅在「以管理员身份」启动的命令行窗口中起效。在 Windows 开始菜单里的搜索栏,输入「cmd」并单击「Enter」键提交,然后鼠标右键单击搜索结果,在弹出的菜单中选择「以管理员身份运行」。以这种方式开启的命令行窗口方可执行上述的「\type{setx}」命令。

验证「\type{D:\foo}」是否被成功添加到系统 \type{PATH} 变量,只需在除 \type{D:\foo} 之外的任一目录验证能否执行 foo.bat,例如

\starttyping
D:\foo> c:
C:\> cd windows\system32  
C:\Windows\System32> foo.bat
不怕命令行  
\stoptyping

\useURL[Windows 命令行][https://www.bilibili.com/video/BV1vk4y1h7LE/]
若不知如何以管理员身份运行命令行窗口,亦可通过图形界面设置 \type{PATH} 变量。我已将上述构建 \type{D:\foo\foo.bat} 以及如何通过图形界面设置系统 \type{PATH} 变量等过程录制为视频,网络链接为\bluebox{\from[Windows 命令行]},藉此避免让层峦叠障的 Windows 窗口截图占据本章太多篇幅。

\section{Linux 终端}

在 Linux 系统中,命令行环境叫作「终端(Terminal)」。终端中可以运行多种 Shell,最为常见的是 Bash Shell。这些 Shell 往往大同小异。终端可以嵌入窗口运行,也可以在没有窗口的情况下运行\footnote{通常情况下,可使用快捷键「\type{Ctrl + Alt + F1~F6}」切换到无窗口的终端。「\type{Ctrl + Alt + F7}」也对应一个终端,它通常被 Linux 窗口系统占用。},我们通常使用的是前者。

Linux 系统发行版众多,每个发行版有其不同的打开终端窗口的方式且因 Linux 用户往往对终端较为熟悉,因而不再赘述如何打开终端窗口。现在,假设终端窗口已经开启。首先,进入 \type{$HOME} 目录,创建子目录 foo:

\starttyping
$ cd $HOME
$ mkdir foo
\stoptyping

\noindent 进入 foo,使用命令输出重定向,将 \type{echo} 命令的输出结果写入 foo.sh 文件:

\starttyping
$ cd foo
$ echo #!/bin/bash > foo.sh
$ echo echo 不怕命令行 >> foo.sh
\stoptyping

\noindent 使用 \type{chmod} 命令为 foo.sh 增加可执行权限,让它像程序一样运行:

\starttyping
$ chmod +x foo.sh
\stoptyping

\noindent 运行 foo.sh:

\starttyping
$ ./foo.sh
不怕命令行
\stoptyping

\noindent 将「\type{$HOME/foo}」添加至系统「\type{PATH}」变量并使之生效:

\starttyping
$ cd $HOME
$ echo 'export PATH=$HOME/foo:$PATH' >> .bashrc
$ source .bashrc
\stoptyping

\noindent 在任一目录下执行 foo.sh 以验证「\type{$HOME/foo}」是否已被添加至「\type{PATH}」变量,例如

\starttyping
$ cd /tmp
$ foo.sh
不怕命令行
\stoptyping

倘若想对 Bash Shell 有更多一些的了解,可以阅读我的拙作「写给高年级小学生的《Bash 指南》」\cite[bash-tutor],它介绍了 Bash 的诸多有趣之处,也许会让你喜欢命令行,而不仅仅是不怕它。

\section{macOS 终端}

\useURL[macOS 终端][https://support.apple.com/zh-cn/guide/terminal/apd5265185d-f365-44cb-8b09-71a064a42125/mac]
我没用过 macOS 系统,在该系统中打开终端窗口,可按 macOS 官方帮助文档介绍的方法进行,详见:\\
\bluebox{{\switchtobodyfont[9.75pt]\from[macOS 终端]}}

至于 macOS 终端的用法,因 macOS 和 Linux 皆为 Unix-like(类 Unix)系统,二者终端环境的用法近乎相同,唯一有些区别的是,从 macOS Catalina 版开始,macOS 默认使用的 Shell 不再是 Bash,而是 zsh。因此,在设置「\type{PATH}」变量时,命令需要变为

\starttyping
$ cd $HOME
$ echo 'export PATH=$HOME/foo:$PATH' >> .zshrc
$ source .zshrc
\stoptyping

\section[installation]{安装 \ConTeXt\ LMTX 最新版本}

安装 \CONTEXT\ LMTX 有两种方式,一种是安装 \CONTEXT\ 开发者提供的 \CONTEXT\ 包,该包仅包含 \CONTEXT\ 环境;另一种是安装 \TEX\ Live 中的 \CONTEXT\ 包。本节介绍前者,下一节介绍后者。

\useURL[installation][https://wiki.contextgarden.net/Installation]
\ConTeXt\ 的最新版本是 \ConTeXt\ LMTX,目前尚在开发和试验阶段,功能虽不稳定,但对于入门而言并无妨碍。从 \bluebox{\from[installation]} 下载对应操作系统类型的 \ConTeXt\ LMTX 安装包,按照该网址的相关介绍进行安装即可,安装过程所需要的全部知识皆已在上文介绍完毕。下文以 Windows 64 位系统为例,给出详细的的安装过程。

\useURL[win-64-version][http://lmtx.pragma-ade.com/install-lmtx/context-win64.zip]
首先,从链接 \bluebox{\from[win-64-version]} 获得面向 Windows 64 位系统的安装包 \type{context-win64.zip} 并假设将其放在 \type{D:\} 解开,所得目录为 \type{context-win64},其结构如下:

\starttyping
D:\context-win64
├── bin
│   ├── mtx-install.lua  
│   ├── mtxrun.exe
│   └── mtxrun.lua
├── installation.pdf
├── install.bat
└── setpath.bat
\stoptyping

将目录 \type{context-win64} 改名为 \type{context},然后打开命令行窗口,依次执行以下命令:

\starttyping
> d:
> cd context
> install.bat
\stoptyping

\noindent 批处理文件 install.bat 能够自动从网络上下载 \ConTeXt\ LMTX 的所有文件,并将其安装在 \type{D:\context} 目录。安装时长取决于网络下载速度。由于服务器在境外,文件下载速度缓慢,可能需要很久方能安装完毕。当安装过程结束后,目前需要再次执行 install.bat,将 \ConTeXt\ LMTX 更新到最新版本。这些都是 \bluebox{\from[installation]} 未告诉我们的。此外,即使安装过程中断,再次运行 install.bat 可继续安装,而不会导致前功尽弃。

验证 \ConTeXt\ LMTX 是否安装成功的方法是,在 \type{D:\context} 目录中执行以下命令:

\starttyping
> setpath.bat
> md test
> cd test
> echo \startTEXpage[frame=on,offset=1pt] > foo.tex
> echo Hello \ConTeXt! >> foo.tex
> echo \stopTEXpage >> foo.tex
> context foo.tex
\stoptyping

\noindent 倘若在 \type{D:\context\test} 目录下能够得到 \type{foo.pdf} 文件,且其内容为 \lower.3em\hbox{\externalfigure[01/foo.pdf]},则意味着已成功安装 \ConTeXt\ LMTX。

最后,将 \type{D:\context\tex\texmf-win64\bin} 添加到系统 \type{PATH} 变量,便可在任一目录使用 \type{context} 命令将扩展名为 \type{.tex} 的文本文件编译成 PDF 文件了。

\section[ctx-in-texlive]{\TEX\ Live 中的 \CONTEXT}

若你熟悉 \TEX\ Live\cite[texlive],推荐使用它提供的 \CONTEXT\ 版本,其功能较为稳定,且国内有镜像网站,在线安装更为便捷。若是新安装 \TEX\ Live,在安装时可直接选择 \type{ConTeXt scheme} 或在 \type{collections} 列表中选择 \type{ConTeXt and packages}。若你的系统中已安装 \TEX\ Live,只是无 \CONTEXT,可使用 TeXLive 包管理器 \type{tlmgr} 安装 \type{collection-context} 即可。

\section{小结}

在一个桌面操作系统中,命令行环境能完成的工作,图形界面通常也能完成,但对于有些任务,命令行环境比图形界面更为高效,反之亦然,原本无需厚此薄彼,但尺有所短,寸有所长,Windows,Linux 和 macOS 也是如此。

\startitemize[packed]
\item Windows 系统的命令行环境,前身是 MS DOS 系统,长期未得进化,现在用起来,捉襟见肘。现代的 PowerShell 更为适用,但因 \ConTeXt\ 目前不能直接在 PowerShell 环境里完成安装,故而上文未曾言及。
\item macOS 无论图形界面还是命令行环境皆胜于 Windows 和 Linux,但其缺点是在不违法的情况下,它只能运行于苹果公司的计算机,费用不菲。
\item Linux 系统的图形界面不及 Windows 和 macOS,但命令行环境优于 Windows\footnote{自 Windows 10 开始,微软在 Windows 系统中构建了 Linux 子系统(WSL),允许用户在 Windows 系统中使用 Linux 命令行环境,但是该环境无法利用 Windows 和 Linux 的图形界面功能。},与 macOS 相当,但 Linux 最显著的特点是,它能帮助每一个使用计算机的人,让他深刻觉察自己以往所宣称的那样热爱自由是否真实。
\stopitemize
