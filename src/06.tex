\chapter{参考文献}

稍微严肃一些的文章,往往会附上一些与文章内容密切相关的参考文献。科研论文更是如此,牛顿都说过自己是站在巨人的肩膀上做事情的。参考文献便是巨人。任何一个 \TeX\ 系统皆不敢对支持参考文献排版这一事宜掉以轻心,否则 \TeX\ 无以为科技文献排版软件之先驱和典范。

\section{\BibTeX}

大多数 \TeX\ 系统在排版参考文献时,皆需依赖 bibtex 程序。由 bibtex 基于文献数据文件生成参考文献列表信息,然后交由 \TeX\ 进行排版,这一过程通常是自动进行的,并不需要用户了解和干预。\ConTeXt\ 现在不需要依赖 bibtex 程序,它已自身内部实现了 bibtex 的全部功能。本文档在谈及 \BibTeX\ 时,主要指其文献数据格式。

\BibTeX\ 文献数据文件是纯文本文件,其中包含着论文、专著和手册等文献数据。例如,一篇期刊论文,其数据格式为

\starttyping
@article{knuth-1984-literate,
  title={Literate programming},
  author={Knuth, Donald Ervin},
  journal={The computer journal},
  volume={27},
  number={2},
  pages={97--111},
  year={1984},
  publisher={Oxford University Press}
}
\stoptyping

\noindent 再例如一本专著,其数据格式为

\starttyping
@book{knuth-1986-texbook,
  title={The texbook},
  author={Knuth, Donald Ervin and Bibby, Duane},
  volume={1993},
  year={1986},
  publisher={Addison-Wesley Reading, MA}
}
\stoptyping

现在,为了学习 \ConTeXt\ 的参考文献排版功能,可将上述文献数据保存为一份文本文件,例如 foo.bib,将该文件作为文献数据文件。

在 foo.bib 相同目录下,建立 \ConTeXt\ 源文件,例如 foo.tex,其内容为

\starttyping[option=TEX]
\usebtxdataset[foo.bib] % 加载 foo.bib
\starttext
Knuth 基于文学编程\cite[knuth-1984-literate]技术开发了
\TeX\ 系统\cite[knuth-1986-texbook]。
\blank
\placelistofpublications
\stoptext
\stoptyping
\noindent 编译 foo.tex,可得以下结果:
\blank[line]
\midaligned{\externalfigure[06/bibtex-example-01.pdf]}
\blank[line]

\section{文献样式}

上一节示例中的文献样式是 \ConTeXt\ 默认样式,除此之外,还有其他三种预定义的文献样式:apa,aps 和 chicago,可在 \type{\placelistofpublications} 之前使用 \type{\usebtxdefinitions} 进行切换。例如使用 aps 样式,

\starttyping[option=TEX]
... ... ...
\usebtxdefinitions[aps]
\placelistofpublications
\stoptyping
\blank[line]
\midaligned{\externalfigure[06/bibtex-example-02.pdf]}
\blank[line]

对于预定义样式,可以进行调整。例如,缩小文献序号后的空白间距,

\starttyping[option=TEX]
\setupbtxlist[apa][distance=.5em,width=fit]
\stoptyping
\blank[line]
\midaligned{\externalfigure[06/bibtex-example-03.pdf]}

\section{自定义文献列表样式}

也许 \ConTeXt\ 提供的参考文献列表样式并不符合你的需求,因此你会忍不住想自己定义一种样式。我的看法是,文献列表样式当由期刊编辑部或专著出版商负责定义,个人无需如此刻意。不过,倘若你正是此类机构工作人员,需要为 \ConTeXt\ 定义符合自己单位要求的文献样式,下文仅能为你提供一个简单的示例,希望有所帮助。

现在,假设我们要定义一种名字叫作 foo 的文献列表样式。首先,按照 \ConTeXt\ 的文件命名约定,建立两份文件,一份是 publ-imp-foo.lua,一份是 publ-imp-foo.tex。在 publ-imp-foo.lua 里写入以下内容:

\starttyping[option=LUA]
local specification = {
    name = "foo",
    categories = {},
}
local categories = specification.categories
categories.article = {
    required = {"author"},
    optional = {
        "title", "journal", "year", "volume", "number", "pages"
    }
}
return specification
\stoptyping

\noindent 上述代码是 Lua 代码,定义了一个表 \type{specification},设定了期刊论文必需和可选元素。事实上,对于定义新的参考文献样式而言,Lua 文件并非必须,只是倘若没有它,BibTeX 数据文件的所有字段皆为可选,而无必需了。

在 publ-imp-foo.tex 文件中写入以下内容

\starttyping[option=TEX]
\definebtx[foo][specification=foo]
\definebtxrendering[foo][specification=foo]
\stoptyping

\noindent 上述代码定义了两个命名空间,第一个用于定义文献的内容结构,第二个用于定义文献的排版样式,它们的文献规格皆为 publ-imp-foo.lua 中定义的名字为 \type{foo} 的表 \type{specification}。这两个命名空间的上层命名空间皆为 \type{btx},而且它们默认会继承 ConTeXt 默认的参考文献内容结构和排版样式的各层子命名空间,亦即我们无需再定义诸如 \type{btx:foo:list},\type{btx:foo:article} 等子命名空间。由于 ConTeXt 参考文献手册中并未对此细致介绍(或者我没有注意到),故而这些内容皆为我的猜测,若有谬误,敬请斧正。

倘若并不清楚自定义的命名空间继承了 ConTeXt 默认的命名空间里的哪些子空间,亦可根据自己需要定义一些子空间。例如,为命名空间 \type{foo} 定义子命名空间 \type{list} 和 \type{list:article}:

\starttyping[option=TEX]
\definebtx[foo:list][foo]
\definebtx[foo:list:article][foo:list]
\stoptyping

现在,为期刊论文设置文献列表样式,在 publ-imp-foo.tex 文件中添加以下内容:

\starttyping[option=TEX]
\startsetups btx:foo:list:article
  % \btxcomma,\btxperiod,\btxcolon,\btxspace
  % 分别为西文的逗号,句号,冒号和空格,用宏是为了醒目。
  \texdefinition{btx:foo:list:author}\btxperiod\btxspace
  \texdefinition{btx:foo:list:title}\btxperiod\btxspace
  \texdefinition{btx:foo:list:journal}[J]\btxcomma\btxspace
  \texdefinition{btx:foo:list:year}\btxcomma\btxspace
  \texdefinition{btx:foo:list:volume}
  \texdefinition{btx:foo:list:number}\btxcolon\btxspace
  \texdefinition{btx:foo:list:pages}\btxperiod
\stopsetups
\stoptyping

\noindent 上述代码定义了期刊论文的构成,其中 \type{\btxcomma},\type{\btxperiod},\type{\btxcolon},\type{\btxspace} 分别为西文的逗号,句号,冒号和空格,用宏是为了醒目。倘若需要使用中文标点,亦可自行定义,例如:

\starttyping[option=TEX]
\def\btxchinesecomma{,}
\def\btxchineseperiod{.}
\def\btxchinesecolon{:}
\stoptyping

在 publ-imp-foo.tex 文件中添加期刊论文各元素的样式定义:

\starttyping[option=TEX]
\starttexdefinition btx:foo:list:author
  \color[blue]{\btxflush{author}}
\stoptexdefinition
\stoptyping

\starttyping[option=TEX]
\starttexdefinition btx:foo:list:title
  \color[red]{\btxflush{title}}
\stoptexdefinition
\stoptyping

\starttyping[option=TEX]
\starttexdefinition btx:foo:list:journal
    \color[magenta]{\btxflush{journal}}
\stoptexdefinition
\stoptyping

\starttyping[option=TEX]
\starttexdefinition btx:foo:list:year
    \btxflush{year}
\stoptexdefinition
\stoptyping

\starttyping[option=TEX]
\starttexdefinition btx:foo:list:volume
    \btxflush{volume}
\stoptexdefinition
\stoptyping

\starttyping[option=TEX]
\starttexdefinition btx:foo:list:number
  \ignorespaces  % 忽略空白字符
  \btxleftparenthesis  % 左括号 (
  \btxflush{number}
  \btxrightparenthesis  % 右括号 )
\stoptexdefinition
\stoptyping

\starttyping[option=TEX]
\starttexdefinition btx:foo:list:pages
  \btxflush{pages}
\stoptexdefinition
\stoptyping

倘若需要调整参考文献列表序号到列表内容之间的距离,并为序号套上方括号,可在 publ-imp-foo.tex 文件中增加以下设定:

\starttyping[option=TEX]
\setupbtxlist[foo][width=auto,distance=1em,startter={[},stopper={]}]
\stoptyping

如果对西方作者的姓名有一些要求,譬如姓氏在前,名字在后且简写,可在 publ-imp-foo.tex 文件中作以下设定:

\starttyping[option=TEX]
\definebtx[foo:list:author][foo:list]
\setupbtx
  [foo:list:author]
  [authorconversion=invertedshort,
   separator:invertedinitials=\btxspace,  % 将名字之后默认的逗号替换为空格
   stopper:initials=\btxspace  % 将简写的名字之后后默认的句点替换为空格
  ]
\stoptyping

最后,在 publ-imp-foo.lua 和 publ-imp-foo.tex 文件所在目录下,建立测试文件 foo.tex 用于验证上述定义参考文献类表样式 foo 是否可用:

\starttyping[option=TEX]
\usebtxdataset[foo] % 加载 foo.bib
\starttext
Knuth 发明了文学编程语言 WEB\cite[knuth-1984-literate]。
\blank
\usebtxdefinitions[foo] % 使用上文定义的参考文献列表样式 foo
\placelistofpublications
\stoptext
\stoptyping

\noindent 结果为

\blank[line]
\midaligned{\externalfigure[06/bibtex-example-04.pdf]}

\section{廉价方案}

倘若你仅仅是在写一份笔记或手册,若半个小时依然未能学会如何为 \ConTeXt\ 定义参考文献样式,我建议随便选一个 \ConTeXt\ 预定义的样式使用。待文章最终定稿后,手工用 \ConTeXt\ 列表将参考文献列表制作成你喜欢的样式,与创作正文内容相比,所费时间通常可忽略不计。例如

\startsample
\def\Cite[#1]{[\in[#1]]}
著名废话学家李某人\Cite[limouren]说过,正经人谁会看参考文献啊!
\startitemize[n,joinedup][left={[},right={]},stopper=]
\item[limouren] 李某人.如何说正确的废话[J].废话学报, 2023, 1(1): 1-10000.
\stopitemize
\stopsample
\typesample
\startblueframedtext
\getsample
\stopblueframedtext

\section{小结}

Hans Hagen 为 \ConTeXt\ 对参考文献的支持专门写了一份约 100 页的手册,见

\starttyping
$ mtxrun --script base --search mkiv-publications.pdf
\stoptyping

\noindent 若有兴致,不妨一睹,特别是在不理解本文自定义文献样式所用的各命令及其参数的含义时,可从该手册获得更多的理解。