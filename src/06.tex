\chapter{参考文献}

稍微严肃一些的文章,往往会附上一些与文章内容密切相关的参考文献。科研论文更是如此,牛顿都说过自己是站在巨人的肩膀上做事情的。参考文献便是巨人。任何一个 \TeX\ 系统皆不敢对支持参考文献排版这一事宜掉以轻心,否则 \TeX\ 无以为科技文献排版软件之先驱和典范。

\section{\BibTeX}

大多数 \TeX\ 系统在排版参考文献时,皆需依赖 bibtex 程序。由 bibtex 基于文献数据文件生成参考文献列表信息,然后交由 \TeX\ 进行排版,这一过程通常是自动进行的,并不需要用户了解和干预。\ConTeXt\ 现在不需要依赖 bibtex 程序,它已自身内部实现了 bibtex 的全部功能。本文档在谈及 \BibTeX 时,主要指其文献数据格式。

\BibTeX\ 文献数据文件是纯文本文件,其中包含着论文、专著和手册等文献数据。例如,一篇期刊论文,其数据格式为

\starttyping
@article{knuth-1984-literate,
  title={Literate programming},
  author={Knuth, Donald Ervin},
  journal={The computer journal},
  volume={27},
  number={2},
  pages={97--111},
  year={1984},
  publisher={Oxford University Press}
}
\stoptyping

\noindent 再例如一本专著,其数据格式为

\starttyping
@book{knuth-1986-texbook,
  title={The texbook},
  author={Knuth, Donald Ervin and Bibby, Duane},
  volume={1993},
  year={1986},
  publisher={Addison-Wesley Reading, MA}
}
\stoptyping

现在,为了学习 \ConTeXt\ 的参考文献排版功能,可将上述文献数据保存为一份文本文件,例如 foo.bib,将该文件作为文献数据文件。

在 foo.bib 相同目录下,建立 \ConTeXt\ 源文件,例如 foo.tex,其内容为

\starttyping[option=TEX]
\usebtxdataset[foo.bib] % 加载 foo.bib
\starttext
Knuth 基于文学编程\cite[knuth-1984-literate]技术开发了
\TeX\ 系统\cite[knuth-1986-texbook]。
\blank
\placelistofpublications
\stoptext
\stoptyping
\noindent 编译 foo.tex,可得以下结果:
\blank[line]
\midaligned{\externalfigure[10-bibtex-example-01.pdf]}
\blank[line]

\section{文献列表样式}

上一节示例中的文献列表是 \ConTeXt\ 默认样式,除此之外,还有其他三种预定义的文献列表样式:apa,aps 和 chicago,可在 \type{\placelistofpublications} 之前使用 \type{\usebtxdefinitions} 进行切换。例如使用 aps 样式,

\starttyping[option=TEX]
... ... ...
\usebtxdefinitions[aps]
\placelistofpublications
\stoptyping
\blank[line]
\midaligned{\externalfigure[10-bibtex-example-02.pdf]}
\blank[line]

对于预定义样式,可以进行调整。例如,缩小文献序号后的空白间距,

\starttyping[option=TEX]
\setupbtxlist[apa][distance=.5em,width=fit]
\stoptyping
\blank[line]
\midaligned{\externalfigure[10-bibtex-example-03.pdf]}

\section{自定义文献列表样式}

也许 \ConTeXt\ 提供的参考文献列表样式并不符合你的需求,因此你会忍不住想自己定义一种样式。对此,我的看法是,文献列表样式当由期刊编辑部或专著出版商负责定义,个人无需如此刻意。不过,倘若你正是此类机构工作人员,需要为 \ConTeXt\ 定义符合自己单位要求的文献样式,下文仅能为你提供一个简单的示例,希望对你有所帮助。

现在,假设我们要定义一种名字叫作 foo 的文献列表样式。首先,需要按照 \ConTeXt\ 的文件命名约定,建立两份文件,一份是 publ-imp-foo.lua,一份是 publ-imp-foo.tex。

在 publ-imp-foo.lua 里写入以下内容:

\starttyping[option=LUA]
local specification = {
  name      = "foo",
  categories = {},
}
local categories = specification.categories
categories.book = {
  required = { "title" },
  optional = {"author", "editor",
              "year",
              "edition", "series", "volume", "number",
              "address", "publisher",
  },
}
return specification
\stoptyping

\noindent 上述代码是 Lua 代码,定义了一个表。\ConTeXt\ 可通过该表明白 \BibTeX\ 文献数据里哪些信息是必须的,哪些信息是可选的。

在 publ-imp-foo.tex 文件中写入以下内容

\starttyping[option=TEX]
\startbtxrenderingdefinitions[foo]
\definebtx
  [foo]
  [default=default,specification=foo]

\definebtxrendering
  [foo]
  [specification=foo,numbering=yes]
\stopbtxrenderingdefinitions
\stoptyping

\noindent 上述代码定义了 foo 环境及其默认样式。此外,\ConTeXt\ 实际上是通过上述代码获知你在 publ-imp-foo.lua 文件里定义的数据表。

下面在 publ-imp-foo.tex 文件为期刊论文定义文献列表样式:

\starttyping[option=TEX]
\startsetups btx:foo:list:article
  \btxdoif {author} {
    \color[blue]{\btxflush{author}}\btxperiod\btxspace
  }
  \btxdoifelse {title} {
     \color[red]{\btxflush{title}}\btxperiod\btxspace
  }{
     No title\btxperiod\btxspace
  }
  \btxdoif {journal} {
    \color[magenta]{\btxflush{journal}}\btxcomma\btxspace
    \btxflush{year},\nospace
    \btxdoifelse {volume} {
      \btxspace
      \btxflush{volume}
      \btxdoif {number} {
        \ignorespaces
        \btxleftparenthesis
        \btxflush{number}
        \btxrightparenthesis
      }
    }{
      \btxdoif {number} {
        \btxlabeltext{default:number}
        \btxspace
        \btxflush{number}
      }
    }
    \btxdoif {pages} {
     \nospace\btxcolon\btxspace\btxflush{pages}
    }
    \btxperiod
  }
  \removeunwantedspaces
\stopsetups
\stoptyping

这是本文档截止至此最为复杂的 \TeX\ 代码,但是所表达的内容非常简单,仅仅是针对期刊论文设定文献列表应当出现作者、文章名称、期刊名称、时间、卷号、期号以及页码等信息。其中,\type{\btxflush} 命令可将 \ConTeXt\ 从 \BibTeX\ 数据格式中解析出的信息输出到排版结果中;\type{\btxcomma},\type{\btxperiod} 之类的命令皆为西文逗号,句号等标点符号;\type{\btxspace} 是西文空格;\type{\btxdoif} 之类的命令是判断 \ConTeXt\ 对 \BibTeX\ 数据格式的解析结果中是否包含某项。

在 publ-imp-foo.lua 和 publ-imp-foo.tex 文件同一目录下,建立测试文件,例如 foo.tex,其内容为

\starttyping[option=TEX]
\usebtxdataset[foo.bib] % 加载 foo.bib
\starttext
Knuth 发明了文学编程语言 WEB\cite[knuth-1984-literate]。
\blank
\usebtxdefinitions[foo]
\placelistofpublications
\stoptext
\stoptyping

用于验证上述定义参考文献类表样式 foo 是否可用。结果为

\blank[line]
\midaligned{\externalfigure[10-bibtex-example-04.pdf]}

\section{小结}

参考文献的样式,我总觉得是学术界拿来唬人的东西。写一篇论文,即使附上几十篇参考文献,用 \ConTeXt\ 列表进行排版,与写文章内容的所耗费的时间相比,完全可以忽略不计。然而在学术界,参考文献如何排版,却成了一门学问。倘若你半个小时依然未能学会如何为 \ConTeXt\ 定义参考文献样式,我建议随便选一个 \ConTeXt\ 预定义的样式使用。待文章最终定稿后,再手工用 \ConTeXt\ 列表将参考文献制作成你需要的样式。