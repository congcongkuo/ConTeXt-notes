\chapter{数学环境}

\TeX\ 原本是 Knuth 专为排版数学类书籍而开发的,亦即排版数学公式是 \TeX\ 最擅长的任务,这也是我对数学的唯一兴趣了。希望你是因为喜欢数学而喜欢 \TeX。

\section{两种模式}

\TeX\ 数学公式有两种模式,一种是正文模式(text mode),一种是显摆模式(display mode),在 \ConTeXt\ 中,自然也是如此。当然,可能你觉得显摆模式这个称谓不够严肃,于是随了它的俗称「行间模式」,并无不可,而且后文我也如此称谓它。

正文模式里的数学公式放在一对美元符号之间,意思是有钱人都精通数学。例如 \type{$a^2 + b^2 = c^2$} 便是正文模式的数学公式,其排版结果为 $a^2 + b^2 = c^2$。\TeX\ 数学公式的行间模式是放在一对双美元符号之间,例如

\starttyping[option=TEX]
$$
\int_0^{+\infty}f(x) {\rm d}x
$$
\stoptyping

\noindent 其排版结果为

$$
\int_0^{+\infty}f(x) {\rm d}x
$$

\noindent 但是,在 \ConTeXt\ 里,行间模式建议用以下形式的语法

\starttyping[option=TEX]
\startformula
\int_0^{+\infty}f(x) {\rm d}x
\stopformula
\stoptyping
\startformula
\int_0^{+\infty}f(x) {\rm d}x
\stopformula

\noindent 至于原因为何,想必你已经知道了,\ConTeXt\ 对 \type{\startformula ... \stopformula} 环境的行间公式默认提供了居中对齐以及前后空白支持。为什么 \ConTeXt\ 不对 \type{$$...$$} 提供类似的支持呢?因为 \ConTeXt\ 需要以一种统一的方式控制行间公式前后的空白大小,例如消除行间公式前后空白,只需

\starttyping[option=TEX]
\setupformulas[spacebefore=0pt,spaceafter=0pt]
\startformula
\int_0^{+\infty}f(x) {\rm d}x
\stopformula
\stoptyping
\setupformulas[spacebefore=0pt,spaceafter=0pt]
\startformula
\int_0^{+\infty}f(x) {\rm d}x
\stopformula

\noindent 若基于 \type{$$...$$} 标记则很难实现该方式。

\section{公式编号}

如同插图和表格,行间数学公式也有一个放置命令 \type{\placeformula}。例如

\starttyping[option=TEX]
\placeformula
\startformula
\int_0^{+\infty}f(x) {\rm d}x
\stopformula
\stoptyping

\placeformula[math-example]
\startformula
\int_0^{+\infty}f(x) {\rm d}x
\stopformula

\ConTeXt\ 行间公式支持引用,例如

\starttyping[option=TEX]
\placeformula[math-example]
\startformula
\int_0^{+\infty}f(x) {\rm d}x
\stopformula
\stoptyping

\noindent 类似插图和表格,使用 \type{\in[...]} 进行引用。例如 \type{\in[math-example]},结果为「\in[math-example]」。

\section{定理和证明}

\ConTeXt\ 未提供可直接用于排版数学定理和证明的命令,但是我们可以借助枚举环境定义它们\cite[theorems]。我们对枚举环境事实上并不陌生,因为早已见识过它的一个特例:列表。

使用 \type{\defineenumeration} 可以定义一种新的枚举特例。例如

\starttyping[option=TEX]
\defineenumeration[theorem][text=定理]
\stoptyping

然后便可使用该特例:

\starttyping[option=TEX]
\starttheorem
凡人皆有一死,凡人皆须侍奉。
\stoptheorem
\stoptyping

\noindent 结果为

\startblueframedtext
\defineenumeration[theorem][text=定理]
\starttheorem
凡人皆有一死,凡人皆须侍奉。
\stoptheorem
\stopblueframedtext

接下来,让定理序号与定理内容的第一行处于同一行,让版面更加紧凑(serried):

\starttyping[option=TEX]
\defineenumeration[theorem][text=定理,alternative=serried]
\starttheorem
凡人皆有一死,凡人皆须侍奉。
\stoptheorem
\stoptyping

\startblueframedtext
\defineenumeration[theorem][text=定理,alternative=serried]
\starttheorem
凡人皆有一死,凡人皆须侍奉。
\stoptheorem
\stopblueframedtext

现在,只需将定理的宽度设为 \type{broad} 或 \type{\textwidth},便可让定理编号和定理内容不至于如此隔阂,顺便将定理内容的字体切换为粗体:

\starttyping[option=TEX]
\defineenumeration[theorem][text=定理,alternative=serried,width=broad,style=\bf]
\starttheorem
凡人皆有一死,凡人皆须侍奉。
\stoptheorem
\stoptyping

\noindent 结果为

\startblueframedtext
\defineenumeration[theorem][text=定理,alternative=serried,width=broad,style=\bf]
\starttheorem
凡人皆有一死,凡人皆须侍奉。
\stoptheorem
\stopblueframedtext

由于枚举环境皆支持引用,因此上述定义的定理可免费继承该功能。例如

\starttyping[option=TEX]
\starttheorem[千面神定理]
凡人皆有一死,凡人皆须侍奉。
\stoptheorem

如定理 \in[千面神定理] 所述……
\stoptyping

\startblueframedtext
\defineenumeration[theorem][text=定理,alternative=serried,width=broad,style=\bf]
\starttheorem[千面神定理]
凡人皆有一死,凡人皆须侍奉。
\stoptheorem

如定理 \in[千面神定理] 所述……
\stopblueframedtext

类似地,也基于枚举环境定义证明,只需去掉序号,并在证明结尾居右放置 $\square$ 符号。例如

\starttyping[option=TEX]
\defineenumeration
  [proof]
  [text=证明,alternative=serried,width=broad,
   number=no,closesymbol={$\square$}]
\startproof
因为人的生命是有限的,为人民服务是无限的。我们应当将有限的生命投入到无限的为人民服务中去。
\stopproof
\stoptyping
\startblueframedtext
\defineenumeration
  [proof]
  [text=证明,alternative=serried,width=broad,
   number=no,closesymbol={$\square$}]
\startproof
因为人的生命是有限的,为人民服务是无限的。我们应当将有限的生命投入到无限的为人民服务中去。
\stopproof
\stopblueframedtext

\section{小结}

现在你已经基本学会了 \ConTeXt\ 数学公式排版。与 \ConTeXt\ 数学排版专家相比,你缺乏的可能主要是如何熟练地输入具体的数学公式。若想在这方面能有所精进,可参考 \ConTeXt\ Wiki 的「Math」页面\cite[math]。