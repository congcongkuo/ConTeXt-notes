% 页面布局
\definepapersize[foo][width=640pt,height=360pt]
\setuppapersize[foo]
\setuplayout[width=middle,topspace=0pt,height=middle,footer=40pt]
\setupbackgrounds[page][backgroundcolor=gray,background=color]
% 字体
\definefallbackfamily
  [myfonts][ss][latinmodernsans][range={0x0000-0x0400},force=yes]
\definefontfamily[myfonts][ss][simhei][bf=simhei,it=kaiti,bi=simhei]
\setscript[hanzi]
\setupbodyfont[myfonts,ss,16pt]
% 常规
\setuppagenumbering[location=]
\setupinterlinespace[line=28pt]
\startuseMPgraphic{square}
  path a, b;
  a := (-.5, 0) -- (-.5, .5) -- (.5, .5);
  b := (-.5, -.5) -- (.5, -.5) -- (.5, 0);
  for i = a, b:
    draw i scaled 10pt withpen pencircle scaled 2pt withcolor darkred;
  endfor;
  draw (0, 0) withpen pensquare scaled 4pt withcolor darkred;
\stopuseMPgraphic
\definesymbol[10][{\lower.15ex\hbox{\useMPgraphic{square}}}]
\setupitemize
  [1][10,packed]
  [distance=0.4em,
   before={\blank[quarterline]},after={\blank[quarterline]}]
% 标题
\setuphead
  [title,chapter]
  [align=center,style=\ssb,color=darkred,
   before={\blank[halfline]},after={\blank[halfline]}]
\setuphead[chapter][conversion=chinesenumerals,alternative=inmargin]

\startuseMPgraphic{processbar}
  numeric w, n, s, t;
  w := \overlaywidth; n := \lastuserpage; s := .5w / (n + 2); t := \userpagenumber - 1;
  path p; p := (fullsquare scaled \overlayheight) randomized 1pt;
  pickup pencircle scaled 1.5pt;
  picture q;
  q := image(for i = 0 upto n - 1:
               p := (p shifted (2s, 0));
               if t = i: fill p withcolor darkgreen; fi;
               draw p withcolor darkgray;%transparent(1, 0.5, darkgray);
             endfor;);
  draw q xsized w;
\stopuseMPgraphic
\defineoverlay[process][\useMPgraphic{processbar}]
\setupfootertexts[{\framed[frame=off,offset=0pt,
                           width=\makeupwidth,height=.8em,
                           background=process,empty=yes]{}}]

\starttext
\startstandardmakeup[align=center]
\ssd \color[darkred]{如何用 \ConTeXt\ 制作幻灯片}
\blank[2*line]
\ssb 李某人
\blank[line]
2023 年 3 月 28 日
\stopstandardmakeup

\startbuffer[foo]
\chapter{蒙}
亨。匪我求童蒙,童蒙求我;初筮告,再三渎,渎则不告。利贞。
\startitemize
\item 初六,发蒙,利用刑人,用说桎梏;以往吝。
\item 九二,包蒙,吉。纳妇,吉;子克家。
\item 六三,勿用取女,见金夫,不有躬,无攸利。
\item 六四,困蒙,吝。
\item 六五,童蒙,吉。
\item 上九,击蒙,不利为寇,利御寇。
\stopitemize
\stopbuffer
\dorecurse{10}{\getbuffer[foo]}
\stoptext
